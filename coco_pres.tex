\documentclass[compress]{beamer}

\useoutertheme[footline=authorinstitutetitle]{miniframes}
\usecolortheme{whale}
\usecolortheme{orchid}
\useinnertheme{rounded}
\usepackage{xcolor}
\usepackage{listings}
\usepackage{placeins}
\usepackage{hyperref}
\usepackage{graphicx}
\lstset{
    language=Python, % choose the language of the code
    basicstyle=\footnotesize\color{black},
    keywordstyle=\color{black}\bfseries, % style for keywords
    commentstyle=\color{blue},
	stringstyle=\color{magenta},
    numbers=none, % where to put the line-numbers
    numberstyle=\tiny, % the size of the fonts that are used for the line-numbers     
    backgroundcolor=\color{white},
    showspaces=false, % show spaces adding particular underscores
    showstringspaces=false, % underline spaces within strings
    showtabs=false, % show tabs within strings adding particular underscores
    frame=single, % adds a frame around the code
    tabsize=2, % sets default tabsize to 2 spaces
    rulesepcolor=\color{gray},
    rulecolor=\color{black},
    captionpos=b, % sets the caption-position to bottom
    breaklines=true, % sets automatic line breaking
    breakatwhitespace=false, 
}



\setbeamerfont{block title}{size={}}

\def\raccoon{
\makebox[\linewidth][c]{\includegraphics[width=70pt]{/home/proycon/Pictures/Local/raccoon.pdf}\FloatBarrier}
}
\def\smallraccoon{
\makebox[\linewidth][c]{\includegraphics[width=30pt]{/home/proycon/Pictures/Local/raccoon.pdf}\FloatBarrier}
}

\newcommand{\myurl}[1]{ { \footnotesize\textcolor{blue}{\url{#1}}  } }

\title{Colibri Core}
\author{Maarten van Gompel \\ 
\emph{Centre for Language Studies} \\ \emph{Radboud University Nijmegen}}
\date{
\begin{center}
\includegraphics[width=15.0mm]{ru-beeldmerk-zwart.eps}
\end{center}
August 28th, 2014}



\begin{document}

\section{Introduction}

\begin{frame}[fragile]
\maketitle
\end{frame}
%\setbeamercolor{background canvas}{bg=black}


%\hypersetup{
%  colorlinks=true,
%  linkcolor=white,
%  urlcolor=blue           % color of external links
%}

\section{Introduction}


\begin{frame}
  \begin{block}{What is colibri-core?}
    \begin{itemize}
     \item Software for the \textbf{extraction}, \textbf{modelling} and
       \textbf{querying} of patterns aka constructions (n-grams, skip-grams, flex-grams)
     \item Low-level (hence the ``core''), more specific software can be built
       on top of it
     \item Highly configurable, many parameters
     \item Designed with memory and speed efficiency in mind, but memory-based
       (lossless)
     \item Binary data formats (quick loading, smaller files)
     \item Three interfaces:
      \begin{itemize}
        \item command-line tools (colibri-patternmodeller,
          colibri-classencode, colibri-classdecode)
        \item C++ API
        \item Python API (binding with native code)
      \end{itemize}
    \end{itemize}
  \end{block}
\end{frame}



\begin{frame}
  \begin{block}{Types of patterns}
    \begin{enumerate}
      \item N-grams: to be or not to be
      \item Skip-grams (one or more fixed-sized gaps): to be {*2*} to be
      \item Flex-grams (one or more gaps of flexibile/undetermined length): to be {*} to be
    \end{enumerate}
  \end{block}
\end{frame}


\begin{frame}
  \begin{block}{Input}
    \begin{enumerate}
      \item Input: plain-text (utf-8), one ``entity'' per line
       (sentence/paragraph/tweet/document), basic conversion from FoLiA supported too.
     \item Input first has to be class-encoded: \texttt{colibri-classencode},
       \texttt{ClassEncoder}.
      \item Output has to be class-decoded: \texttt{colibri-classdecode},
        \texttt{ClassDecoder}
    \end{enumerate}
  \end{block}
\end{frame}

\begin{frame}

  \begin{block}{Compression through class encoding}
    \begin{itemize}
      \item All word types in a corpus are ranked (frequent to rare)
      \item Each word is encoded by its rank/class number
      \item Variable-width encoding: Frequent words (high up in the ranking)
        take less bytes to represent than rare word types
      \item Results in roughly $50\%$ compression
      \item Integer comparison (by token) much quicker than string comparison
        (by character)
      \item On large data, Colibri-core will be faster and smaller than naive string-based representation
      \item Stored on disk in binary format for quick loading
      \item Caveat: Corpora are only comparable if they use the same class
        encoding!
    \end{itemize}
  \end{block}

\end{frame}

\begin{frame}
  \begin{block}{Extraction}
    \begin{itemize}
      \item What patterns are in a corpus? 
      \item How many times do they occur?
      \item Where do they occur in the corpus? (= Indexed modelling)
      \item Extraction constrained by another model (train/test paradigm)
    \end{itemize}
    Result of such extraction is a \textbf{Pattern Model}
  \end{block}
\end{frame}

\begin{frame}
  \begin{block}{Modelling}
    \begin{itemize}
      \item How many patterns are there?
      \begin{itemize}
        \item for a specific pattern
        \item of a particular pattern type and/or length?
        \item how much of the corpus is covered?
        \item basic statistics
      \end{itemize}
      \item How do patterns relate to others?
      \begin{itemize}
        \item What patterns subsume others? (parent/child relations)  What fits
          in a gap of a skipgram?
        \item What patterns follow each-other in the text? 
        \item What patterns co-occur, how often? (Jaccard, nPMI) 
      \end{itemize}
      \item How do two corpora compare?
      \begin{itemize}
        \item How many/which patterns of the training corpus occur in the
          test corpus as well?
      \end{itemize}
      \item Reverse indexing
      \begin{itemize}
          \item What patterns can be found for a certain position in the corpus?
          \item Easy context extraction
      \end{itemize}
    \end{itemize}
  \end{block}
\end{frame}



\begin{frame}

  \begin{block}{High-level Data structures}
    \begin{itemize}
      \item \texttt{Pattern}: Encoded representation of a pattern, the core
        building block
      \item \texttt{PatternModel}: ${pattern} \mapsto {occurrencecount}$
      \begin{itemize}
        \item \texttt{IndexedPatternModel} ${pattern} \mapsto {indexreferences}$
      \end{itemize}
      \item \texttt{PatternAlignmentModel}:  ${pattern}_1 \mapsto {pattern}_2 \mapsto {value(s)}$
    \end{itemize}
  \end{block}

\end{frame}


\begin{frame}

  \begin{block}{Low-level Data structures}
     C++ library is set up in such a way (templating) that you can exchange the
      underlying data structure whilst maintaining the same interface

    \begin{itemize}
      \item \texttt{PatternStore} \emph{(abstract)}
      \begin{itemize}
        \item \texttt{PatternMap}: Hash-map \emph{(C++11 unordered\_map)}:
          ${pattern} \mapsto {value(s)}$
        \item \texttt{PatternSet}: Set \emph{(C++11 unordered\_set)}
      \end{itemize}
      \item \texttt{IndexedCorpus}: ${indexreference} \mapsto {unigram}$
      \begin{itemize}
        \item \ldots as \textbf{Reverse Index} with a Pattern Model: ${indexreference} \mapsto {patterns}$
      \end{itemize}
    \end{itemize}
  \end{block}

\end{frame}



\begin{frame}
  \begin{block}{Open-ended}
    \begin{itemize}
      \item Colibri-core is just the foundation for more specific tools
      \item Lots of possible use-cases
      \item Things built on top of colibri core:
      \begin{itemize}
        \item \textbf{Colibri-MT}: Modelling of source-side context for Machine
          Translation, through classifiers
        \begin{itemize}
          \item \textbf{Colibrita}: Modelling of L2 context for L1 fragments, SemEval
            2014 Task 5, through classifiers
        \end{itemize}
        \item \textbf{cococpyp}: Nonparametric Bayesian modeling with Pitman-Yor process priors
        \item \textbf{DEvents} event pair extraction
      \end{itemize}
    \end{itemize}
  \end{block}
\end{frame}






\section{The End}

\begin{frame}

\raccoon

\begin{center}

\medskip

\Large{Questions?}


\end{center}

\end{frame}



\end{document}
