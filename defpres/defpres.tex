\documentclass[8pt]{beamer}
\usepackage[dutch]{babel}
\usepackage{lmodern}
\usetheme{Madrid}


% Change base colour beamer@blendedblue (originally RGB: 0.2,0.2,0.7)
\colorlet{beamer@blendedblue}{green!40!black}

\title{\textbf{Co}ntext as \textbf{Li}nguistic \textbf{Bri}dges}
\author{Maarten van Gompel}
\date{\vspace{-10ex} {24 maart 2020}}
\usepackage{graphicx}
\usepackage{placeins}



\def\raccoon{
\makebox[\linewidth][c]{\includegraphics[width=70pt]{/home/proycon/Pictures/All/raccoon.pdf}\FloatBarrier}
}
\def\smallraccoon{
\makebox[\linewidth][c]{\includegraphics[width=30pt]{/home/proycon/Pictures/All/raccoon.pdf}\FloatBarrier}
}

\begin{document}

\begin{frame}
	\titlepage
    \makebox[\linewidth][c]{\includegraphics[width=6cm]{../thesis/drawing-eps-converted-to.pdf}\FloatBarrier}
\end{frame}

\section{Inleiding}

\begin{frame}{Inleiding}

	\begin{block}{Context}
        \begin{itemize}
            %Algemene intuitie
            \item \textbf{Intuïtie:} \emph{``Context speelt een belangrijke rol in taal''}
            \item Betekenis van woorden, zinsneden, zinnen etc... is afhankelijk van de context waarin ze verschijnen
            \item Context helpt eventuele ambiguïteit in betekenis op te lossen
        \end{itemize}
	\end{block}

    \begin{block}{Voorbeeld: een woord}
        {\Large\textbf{``bank''}}

        Volgens van Dale:
        \begin{enumerate}
            \item zitmeubel voor meer dan één persoon
            \item verkooptafel: toonbank
            \item werktafel: draaibank
            \item door de natuur gevormde ondiepte
            \item instelling die gelden beheert en uitleent
            \item (bij kansspelen) inzet van de hoofdspeler tegen alle andere samen
            \item centrale bewaar- en uitwisselingsplaats; = opslagplaats: bloedbank, vacaturebank
        \end{enumerate}
    \end{block}
\end{frame}

\begin{frame}

    \begin{block}{Voorbeeld 1: een woord in context}
        {\Large\textbf{``bank''}}
        \begin{itemize}
            \item \emph{``De \textbf{bank} heeft al mijn geld verloren.''}
            \item \emph{``De hond mag op de \textbf{bank} liggen.''}
        \end{itemize}
    \end{block}

    \begin{block}{Word Sense Disambiguation}
        \begin{itemize}
            \item Context helpt bij het vinden van de betekenis
            \item Het automatisch disambigueren van de juiste betekenis van een woord heet \textbf{Word Sense Disambiguation}.
        \end{itemize}
    \end{block}
\end{frame}


\begin{frame}

    \begin{block}{Voorbeeld 1: Een woord in context}
        {\Large\textbf{``bank''}}
        \begin{itemize}
            \item \emph{``De \textbf{bank} heeft al mijn geld verloren.''}
            \begin{itemize}
                \item<2-> \emph{``La \textbf{banque} à perdu tout mon argent.''}
            \end{itemize}
            \item \emph{``De hond mag op de \textbf{bank} liggen.''}
            \begin{itemize}
                \item<2-> \emph{``Le chien peut se mettre sur \textbf{le canapé}.''}
            \end{itemize}
        \end{itemize}
    \end{block}

    \begin{block}{Betekenis en Vertalingen}
        \begin{itemize}
            \item Het vinden van de juiste betekenis is belangrijk bij vertalen
            \item Een woord kan anders vertalen naar gelang de betekenis
        \end{itemize}
    \end{block}


\end{frame}

\begin{frame}

    \begin{block}{Voorbeeld 2: Een woord in context}
        {\Large\textbf{``tempo''}} (portugees)
        \begin{itemize}
            \item \emph{``Faz bom \textbf{tempo} hoje.``}
            \begin{itemize}
                \item<2-> \emph{``Het is lekker \textbf{weer} vandaag``}
            \end{itemize}
            \item \emph{``Ele tem \textbf{tempo} para mim.''}
            \begin{itemize}
                \item<2-> \emph{``Hij heeft \textbf{tijd} voor mij.``}
            \end{itemize}
        \end{itemize}
    \end{block}

\end{frame}

\begin{frame}

	\begin{block}{Onderzoek}
        \begin{itemize}
            \item test

        \end{itemize}
	\end{block}

\end{frame}

\end{document}

