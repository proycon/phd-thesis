\documentclass[8pt]{beamer}
\usepackage[dutch]{babel}
\usepackage{lmodern}
\usetheme{Madrid}

\usepackage{url}
\usepackage{natbib}

\usepackage{amsmath}
\usepackage{amsopn}
\usepackage{amsthm}

\DeclareMathOperator*{\argmin}{arg\,min}
\DeclareMathOperator*{\argmax}{arg\,max}


% Change base colour beamer@blendedblue (originally RGB: 0.2,0.2,0.7)
\colorlet{beamer@blendedblue}{green!40!black}

\title{\textbf{Co}ntext as \textbf{Li}nguistic \textbf{Bri}dges}
\author{Maarten van Gompel}
\date{\vspace{-10ex} {24 maart 2020}}
\usepackage{graphicx}
\usepackage{placeins}



\def\raccoon{
\makebox[\linewidth][c]{\includegraphics[width=70pt]{/home/proycon/Pictures/All/raccoon.pdf}\FloatBarrier}
}
\def\smallraccoon{
\makebox[\linewidth][c]{\includegraphics[width=30pt]{/home/proycon/Pictures/All/raccoon.pdf}\FloatBarrier}
}

\begin{document}

\begin{frame}
	\titlepage
    \makebox[\linewidth][c]{\includegraphics[width=6cm]{../thesis/drawing-eps-converted-to.pdf}\FloatBarrier}
\end{frame}

\section{Inleiding}

\begin{frame}{Inleiding}

	\begin{block}{Context}
        \begin{itemize}
            %Algemene intuitie
            \item \textbf{Intuïtie:} \emph{``Context speelt een belangrijke rol in taal''}
            \item Betekenis van woorden, zinsneden, zinnen etc... is afhankelijk van de context waarin ze verschijnen.
            \item Context helpt eventuele ambiguïteit in betekenis op te lossen
        \end{itemize}
	\end{block}

    \begin{block}{Voorbeeld: een woord}
        {\Large\textbf{``bank''}}

        Volgens van Dale:
        \begin{enumerate}
            \item zitmeubel voor meer dan één persoon
            \item verkooptafel: toonbank
            \item werktafel: draaibank
            \item door de natuur gevormde ondiepte
            \item instelling die gelden beheert en uitleent
            \item (bij kansspelen) inzet van de hoofdspeler tegen alle andere samen
            \item centrale bewaar- en uitwisselingsplaats; = opslagplaats: bloedbank, vacaturebank
        \end{enumerate}
    \end{block}
\end{frame}

\begin{frame}

    \begin{block}{Voorbeeld 1: een woord in context}
        {\Large\textbf{``bank''}}
        \begin{itemize}
            \item \emph{``De \textbf{bank} heeft al mijn geld verloren.''}
            \item \emph{``De hond mag op de \textbf{bank} liggen.''}
        \end{itemize}
    \end{block}

    \begin{block}{Word Sense Disambiguation}
        \begin{itemize}
            \item Context helpt bij het vinden van de betekenis
            \item Het automatisch disambigueren van de juiste betekenis van een woord heet \textbf{Word Sense Disambiguation}.
        \end{itemize}
    \end{block}
\end{frame}


\begin{frame}{Inleiding}

    \begin{block}{Voorbeeld 1: Een woord in context}
        {\Large\textbf{``bank''}}
        \begin{itemize}
            \item \emph{``De \textbf{bank} heeft al mijn geld verloren.''}
            \begin{itemize}
                \item \emph{``La \textbf{banque} à perdu tout mon argent.''}
            \end{itemize}
            \item \emph{``De hond mag op de \textbf{bank} liggen.''}
            \begin{itemize}
                \item \emph{``Le chien peut se mettre sur \textbf{le canapé}.''}
            \end{itemize}
        \end{itemize}
    \end{block}

    \begin{block}{Betekenis en Vertalingen}
        \begin{itemize}
            \item Het vinden van de juiste betekenis is belangrijk bij vertalen
            \item Een woord kan anders vertalen naar gelang de betekenis
        \end{itemize}
    \end{block}


\end{frame}

\begin{frame}{Inleiding}

    \begin{block}{Voorbeeld 2: Een woord in context}
        {\Large\textbf{``tempo''}} (portugees)
        \begin{itemize}
            \item \emph{``Faz bom \textbf{tempo} hoje.``}
            \begin{itemize}
                \item \emph{``Het is lekker \textbf{weer} vandaag``}
            \end{itemize}
            \item \emph{``Ele tem \textbf{tempo} para mim.''}
            \begin{itemize}
                \item \emph{``Hij heeft \textbf{tijd} voor mij.``}
            \end{itemize}
        \end{itemize}
    \end{block}

\end{frame}

\begin{frame}{Inleiding}

	\begin{block}{Onderzoek}
        \begin{itemize}
            \item De rol van contextinformatie in automatische vertaling
            \item \textbf{Hoofdvraag:} In hoeverre kunnen we automatische vertaling verbeteren door contextinformatie uit de brontaal mee te nemen?
            \item Empirische studie: Schrijven van software, trainen op trainingsdata, testen op testdata,
                vergelijken met anderen, conclusies afleiden
        \end{itemize}
	\end{block}

\end{frame}


\section{Cross-Lingual Word Sense Disambiguation (hfd 3)}

\begin{frame}{Cross-Lingual Word Sense Disambiguation (hfd 3)}


	\begin{block}{Aanpak WSD}
        \begin{itemize}
            \item We bouwen een \textbf{classifier-gebaseerd} cross-lingual WSD systeem in navolging van eerdere studies \citep{Hoste+02,Hendrickx+02}
            \begin{itemize}
                \item \color{teal} memory-based learning / TiMBL software / IB1 algoritme
            \end{itemize}
            \item Het systeem leert op basis van voorbeelden hoe een woord met bepaalde \textbf{contextuele kenmerken}
                vertaald wordt.
            \begin{itemize}
                \item \color{teal} Lokale contextinformatie (woordvorm): $x$ woorden links, $y$ woorden rechts
                \item Linguïstisch verrijkte lokale contextinformatie (PoS$+$lemma)
                \item Globale contextinformatie (woordvorm) \footnotesize{(binary bag of word features)}
            \end{itemize}
            \item Deelname aan drie SemEval taken
            \begin{itemize}
                \item SemEval 2010: Cross Lingual Lexical Substitution {\footnotesize(Engels naar Spaans)} \citep{CLLS}
                \item SemEval 2010: Cross Lingual Word Sense Disambiguation {\footnotesize(Engels naar Nederlands/Spaans)} \citep{WSD}
                \item SemEval 2013: Cross Lingual Word Sense Disambiguation {\footnotesize(Engels naar
                    Nederlands/Spaans/Frans/Duits/Italiaans)} \citep{CLWSD2013TASKPAPER}
            \end{itemize}
        \end{itemize}
	\end{block}

\end{frame}


\begin{frame}{Cross-Lingual Word Sense Disambiguation (hfd 3)}
    \begin{block}{Deelvragen \& Resultaten}
        \begin{enumerate}
            \item Welke informatie draagt het meeste bij tot een juiste vertaling?
            \begin{itemize}
                \item \color{teal} Linguïstisch-geïnformeerd of niet?
                \begin{itemize}
                    \item \color{teal} Lemmas features helpen, PoS juist niet
                \end{itemize}
                \item Lokale en/of globale features?
                \begin{itemize}
                    \item \color{teal} Globale features vallen tegen ondanks wat eerdere succesen
                \end{itemize}
                \item Contextgrootte?
                \begin{itemize}
                    \item \color{teal} Kleine contextgrootte werkt het best (1 links, 1 rechts)
                \end{itemize}
            \end{itemize}
            \item Welke optimalisatietechnieken kunnen we toepassen?
            \begin{itemize}
                \item \color{teal} Arbiter Voting (WSD1, 2010)
                \item \color{teal} Automatische feature selectie (WSD2, 2013)
                \item \color{teal} Classifier parameteroptimalisatie (WSD2, 2013)
            \end{itemize}
        \end{enumerate}
    \end{block}
    \begin{block}{Competitie Resultaten}
        \begin{itemize}
            \item Winnende scores in de Cross Lingual WSD taak (2010), later ook een aantal winnende scores in 2013
        \end{itemize}
    \end{block}
\end{frame}

\section{Vertalen van L1 fragmenten in een L2 context}

\begin{frame}{Vertalen van L1 fragmenten in een L2 context (hfd 4-6)}
    \begin{block}{Een nieuwe insteek}
        \begin{enumerate}
            \item kunnen we fragmenten automatisch vertalen met een \textbf{anderstalige context}?
            \begin{itemize}
            \item het gebruiken van korte fragmenten van de ene taal in een anderstalige context (bv. zin)
            \end{itemize}
            \item \textbf{codewisseling/code switching}: het overspringen tussen twee talen
        \end{enumerate}
    \end{block}

    \begin{block}{Voorbeeld 1: Codewisseling met één woord}
        L1 = Nederlands, L2 = Duits
        \begin{enumerate}
            \item ''Alles was er sagt ist \textbf{altijd} falsch''
            \item ''Alles was er sagt is \textbf{immer} falsch''
        \end{enumerate}
    \end{block}

    \begin{block}{Voorbeeld 2: Codewisseling met langer fragment}
        L1 = Nederlands, L2 = Duits
        \begin{enumerate}
            \item ''Alles \textbf{wat hij zegt} ist immer falsch''
            \item ''Alles \textbf{was er sagt} is immer falsch''
        \end{enumerate}
    \end{block}
\end{frame}

\begin{frame}{Vertalen van L1 fragmenten in een L2 context (hfd 4-6)}
    \begin{block}{Opzet}
        \begin{itemize}
            \item Zit er waarde in dit idee?
            \begin{itemize}
                \item Ja, denk aan vertaalhulp systemen
            \end{itemize}
            \item Kunnen we dit net als voorheen met classifiers oplossen? $\rightarrow$ \textbf{pilot study} (hfd 4)
            \begin{itemize}
                \item Ja
                \item Maar; we hebben geen echte representatieve data om op te testen
            \end{itemize}
            \item Testdata samengesteld en daarmee een nieuwe SemEval taak organizeren (hfd 5)
            \begin{itemize}
                \item 4 taalparen (L1-L2): Engels-Spaans, Engels-Duits, Frans-Engels, Nederlands-Engels
                \item Handmatige dataverzameling uit verschillende bronnen
                \item 6 kandidaten doen mee met onze taak
            \end{itemize}
            \item Met de nieuwe testdata ons eigen systeem \emph{(colibrita)} opnieuw testen en verbeteren (hfd 6)
        \end{itemize}
    \end{block}
\end{frame}

\section{Statistical Machine Translation}

\begin{frame}{Statistical Machine Translation (hfd 6-7)}

    \begin{block}{Van WSD naar SMT}
        Van \textbf{Word Sense Disambiguation} naar (phrase-based) \textbf{Statistical Machine Translation}
        \begin{itemize}
            \item Naar mate we langere fragmenten bezien begeven we ons steeds meer op de het terrein van de SMT
            \item Kunnen SMT technieken ook toegepast worden in het code-switching scenario?
            \begin{itemize}
                \item Ja, twee deelnemers aan onze taak laten dit duidelijk zien
                \item Resultaten beter dan pure classifier aanpak
            \end{itemize}
            \item \textbf{Beste van beide werelden?} Kunnen we de expliciete modellering van de classifiers integreren in SMT en heeft dat meerwaarde? (hfd 6,7)
           \begin{itemize}
               \item We vergelijken classifiers, classifiers geïntegreerd in SMT, en pure SMT
           \end{itemize}

        \end{itemize}
    \end{block}
\end{frame}

\begin{frame}{Statistical Machine Translation (hfd 6-7)}
    \begin{block}{Integratie van classifiers in SMT}
        \begin{itemize}
            \item Op welke manieren kunnen we classifiers in SMT integreren?
            \begin{itemize}
                \item \color{teal} \footnotesize{(\emph{classifiertype:} classifier experts vs. monolithisch) , (\emph{weegmethode:} replace vs append)}
                \item Minimale verschillen met weinig effect in het algemeen, geen eenduidige conclusie te vormen
            \end{itemize}
            \item Heeft data zonder linguïstische verrijking een maarwaarde in deze opstelling?
            \begin{itemize}
                \item Nee, \color{teal} de \textbf{hoofdconclusie} is dat we expliciet proberen te modelleren wat de bestaande modellen
                    (vertaalmodel + taalmodel) impliciet al goed genoeg dekken. Daar waar contextinformatie uit de
                    brontaal een disambiguerende rol kan spelen, is hij al potentieel beschikbaar in het vertaalmodel.
           \end{itemize}
        \end{itemize}
    \end{block}
\end{frame}

\begin{frame}{Conclusies}
    \begin{block}{Conclusies}
        \begin{enumerate}
            \item Contextinformatie is nuttig voor vertaling (ook met anderstalige context)
            \item We sluiten een lijn van onderzoek
                \citep{Stroppa+07,Rejwanul+11}
                {\color{teal} \footnotesize (de integratie van niet-linguïstische-geinformeerde
                classifier-gebaseerde WSD technieken in (PB)SMT)}
                en concluderen dat deze integratie geen meerwaarde
                biedt.
                %meer voor de hand ligt het LM verbeteren
            \item \color{teal} We creëren vaak marginale lokale verschillen en concluderen dat een gedegen evaluatie op automatische
                metrieken en bij een gebrek aan meerdere referentievertalingen erg lastig is.
            \item \color{teal} We benadrukken het belang van testen op meerdere datasets, meerdere taalparen, en het doen van
                significantietests om niet te snel conclusies te trekken.
        \end{enumerate}
    \end{block}
\end{frame}

\begin{frame}[allowframebreaks]
        \frametitle{References}
        \bibliographystyle{apalike}
        \bibliography{../thesis/thesis}
\end{frame}

\end{document}

