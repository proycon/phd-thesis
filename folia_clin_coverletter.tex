- stylistic and typo fixes as recommended by the reviewers




> In the last paragraph of section 3.5 Metadata, you mention that the
> distinction between data and metadata is fuzzy. To me, there is a
> clear distinction. Could you clarify?

I was referring mainly to the notions of 'annotator' and 'annotatortype' being arguably a kind of metadata, but I agree this point may be weak and unnecessary. I removed it entirely.

> This article aims to give a general description of FoLiA, summing up
> its features. However, as far as I can see, there is no description of
> the FoLiA schema used. It would be nice to find out in what kind of
> schema has been used, and to what extent the validation can help the
> user to better format XML documents, keeping in mind the flexibility
> the FoLiA developers have in mind.

There is a short reference to a RelaxNG schema for validation. I added another
short reference at the beginning of section 3 and refer back to it in section
3.1 where I now say a bit more about validation. 

> In Figure 3, an example of the offset structure is given, but the
> notion of "relative" offsets is not explained. As far as I can see,
> there is an offset error in element xml:id="example.p.1.s.1.w.3, where
> the offset of "Mr." should be 8; or did I get it wrong?

Very true, there was an error

> Section 3.2: does "offset" refer to character or word offset? Describe
> what you mean with "text" and "start" (especially if "offset" refers to
> character offset).

I rephrased this explanation to make it more clear.

> Sections 3.4 and 3.6 are quite short and could be expanded or incorporated
> in other sections.

I merged 3.6 (Set definition) and part of 3.4 (Alternatives & Corrections) into
section 3.3 (Sets, classes and common attributes). The other part of 3.4 got
merged into the section on Higher Order Annotation (previously 3.7)

> Section 3.7 is unclear. Are the concepts "set" and "subset" the best way
> to describe the difference between annotation and features?

I have tried to add some extra explanation to clarify the point, although I
find it hard to see what part in particular causes confusion. 
We do think the notion "subset" is an adequate way to introduce higher-order
features in FoLiA.





