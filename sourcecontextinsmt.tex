\documentclass[11pt]{article}
%\usepackage{acl2014}
\usepackage{times}
\usepackage{url}
\usepackage{latexsym}
\special{papersize=210mm,297mm} % to avoid having to use "-t a4" with dvips 
%\setlength\titlebox{6.5cm}  % You can expand the title box if you really have to
\usepackage{graphicx}
\usepackage{placeins}
\usepackage{framed}
\usepackage{pbox}
\usepackage{supertabular}
\usepackage{listings}
\usepackage[utf8]{inputenc}
\usepackage{gb4e}

\title{Classifier-based modelling of source-side context information for Statistical Machine Translation}


\author{Maarten van Gompel \& Antal van den Bosch \\
 Centre for Language Studies \\
  Radboud University Nijmegen \\
  {\tt proycon@anaproy.nl}}

%\date{}


\begin{document}
\maketitle

\begin{abstract} 
We present in-depth research into the modelling of source-side
context to improve Phrase-based Statistical Machine Translation. Statistical
Machine Translation systems typically consist of a translation model and a
language model. The former maps phrases in the source language to the target
language, without regard for the context in which the source phrases occur. The
latter models just the target language, and acts as a target-side model of
context information after translation. We attempt to independently reproduce a
line of existing research and test whether considering context information
directly in the translation model has a positive effect on translation quality.
We furthermore investigate various ways classifier-based models can be
integrated into Statical Machine Translation.  We will use proven techniques
from Word Sense Disambiguation, effectively integrating WSD techniques in
Statistical Machine Translation. Our approach is classifier-based and our focus
is exclusively unsupervised, we therefore do not include any additional linguistic
features. 
\end{abstract}

\section{Introduction}

In Phrase-based Statistical Machine Translation (SMT) the problem of
translating an input sentence from a source language to a target language is
perceived as a game of probabilities and a search for the most probable
translation option.  These probabilities are expressed in a number of models
that specialize in a certain aspect relevant to the translation process. The
``phrase-based'' characteristic is due to phrases being the building blocks of
the translation model. This can be contrasted to earlier approaches in
Statistical Machine Translation which started as word-based \cite{OCHNEY?}.
Phrases in this sense have to be perceived simply one or more words, i.e.
n-grams of variable length (including unigrams). Moreover, they are not at all
required to form a proper linguistic constituent of any kind.

Two models are at the core of phrase-based SMT: first there is the translation
model which maps the translation phrases in the source language ($s$) to
phrases in target language ($t$), this mapping is expressed as a vector of
probabilities, most notably $P({phrase}_s|{phrase}_t)$ and
$P({phrase}_t|{phrase}_s)$. This component can be seen to model the notion of
``semantic faithfullness''; if you translate a phrase from one language to
another, you want the meaning to be preserved as accurately as possible. The
second core model is the language model, this model in monolingual in nature
and models \emph{the target language}. It models what words are likely to
follow others and can be interpreted as modelling the ``syntactic fluency''
notion of translation; a translation should be in a natural word-order and
sound natural. A Machine Translation \emph{decoder} optimises a log-linear
model of these two, and additional other, models. Given an input sentence in
the source language, it searches through a vast space of all ``possible''
translation options, most non-sensical, for a path maximising the probabilities
according to each of the models, taking into account different weights they may
be assigned.

The study we currently present focusses on the role of surface-form context
information in this SMT process. The Language Model effectively models context
for the target language, it makes sure that a translated phrase fits nicely
along other translated phrases. But in SMT there is no component modelling
context for the source-language, whereas intuitively source-side context may
provide a powerful cue for translation. Consider the word ``bank'' and its
Spanish translation in examples~\ref{ex:bank1} and \ref{ex:bank2}.

\begin{exe} %gb4e package
\ex \textbf{English:} I don't trust the bank. \\
    \textbf{Spanish:} No me fio del banco.
\label{ex:bank1}

\ex \textbf{English:} The boat headed towards the bank of the river. \\
    \textbf{Spanish:} El barco se dirigió hacia la orilla del río.
\label{ex:bank2}
\end{exe}

The same English word, ``bank'' may express multiple semantic senses, some of
which are expressed by different words in Spanish. Source-side context
information may provide valuable clues to what sense is being employed, and
therefore what translation is correct.  The words ``boat'' and the phrase ``of
the river'' in example \ref{ex:bank2} make it pretty clear that we are using
bank in its maritime sense. Example \ref{ex:bank1} is less obvious, but the
word ``trust'' could be seen to be a cue when the noun ``bank'' denotes a
financial institution.

These examples are meant to illustrate the intuition that is behind our
research hypothesis. We hypothesise that the inclusion of source-side context
information in the translation model improves translation results, as the
context helps in providing a more accurate disambiguation. A counter-hypothesis
to this would be that whilst source-side context information is not modelled
explicitly, it is implicitly captured by the combination of translation model
and language model, and explicit modelling has no added value.

There is clear and obvious overlap between what we do here and the field of
Word Sense Disambiguation (WSD). We effectively test an integration of proven
techniques from WSD in Statistical Machine Translation, and apply these to
phrases rather than just words.

WSD systems often employ a variety of linguistic features. The focus of our
study, however, is fully unsupervised. We are interested only in the surface
forms, the text as-is, and stay as close as possible to vanilla Phrase-Based
Statistical Machine Translation, without using any language-specific external
resources. In this fashion, we attempt to assess the merit of source-side
context information as it is in its unmodified form. 

Not introducing extra data for the translation system means our goals have to
be set more modest as well. We do not expect as much gains as 


\section{Previous research}

The idea to integrate WSD approaches in SMT is not new, nor is the idea to use
source-side context information to disambiguate in translation. Various studies
have been conducted with mixed results. In the early days of SMT,
\cite{GarciaVarea+02} already explicitly modelled source-side context in a
maximum entropy model for word-based SMT, and report slightly improved error
rates on a translation task.

\cite{CarpuatWu05} were the first to directly tackle the question whether
full-scale WSD models were beneficial to translation quality when integrated in
SMT systems, and thus their work forms important foundation for our own study.
Their approach uses an ensemble classification model that integrates
position-sensitive, syntactic, and local collocational features, which has
proven itself in competitive WSD tasks. This includes linguistic features such
as part-of-speech tags and lemmas, as well as more complex syntactic relations.
They test for a single Chinese-to-English test-set only, and only use BLEU,
which raises some questions on whether their conclusions would hold on
different language pairs, test sets, and using different evaluation metrics.

\cite{CarpuatWu05} place strong focus on the WSD model rather than the SMT
model, whereas we place more focus on the SMT model and the integration method,
and keep the ``WSD-model'' relatively simple. Furthermore, the method they
employ a simpler word-based form of Statistical Machine Translation and the
level of integration seems limited.

Despite their efforts, they reach the surprising conclusion that inclusion of
WSD techniques does \emph{not} yield better translation quality. Will these
results hold in a more modern Phrase-based Statistical Machine Translation
approach?

Two years later they expanded their study to full phrasal units
\citep{CarpuatWu07} and, for the first time, found results that did support the
hypothesis that SMT benefits from the integration of WSD techniques. They now
focus on better integration in \emph{phrase-based} SMT: ``Rather than using a
generic SenseEval model as we did in \cite{CarpuatWu05}, here both the WSD
training and the WSD predictions are integrated into the phrase-based SMT
framework.'' \citep{CarpuatWu07}. They also broaden their use of evaluation
metrics, yet still test on only Chinese to English.

The work of \cite{Gimenez+07} is similar, they use support vector machines to
predict the phrase translation probabilities for the phrase-translation table
component of SMT, rather than relying on the context-unaware Maximum Likehood
Estimate the statistical process produces. The feature vector for their
classifiers consists of both local context as well as global context features.
In addition to the surface forms of the words, they do rely on shallow
linguistic features such as Part-of-Speech tags and lemmas. They conduct a
manual evaluation judged on fluency and adequacy, and conclude that considering
context improves adequacy, yet does not benefit fluency. They remark that the
integration of the classifier probabilities in an SMT framework needs further
study, which is something that will indeed be a focus in our present study.

The year 2007 saw a culmination of various studies integrating WSD techiques in
SMT using classifiers. A third study in this trend was \cite{Stroppa+07}. They
have a strong focus on the word form, as does this present study, and add only
part-of-speech features. On IWSLT 2006 data for Chinese-English and
Italian-English, they achieve a significant improvement for the former, whereas
the BLEU score for the latter fails to pass the significance test. We will
attempt to reproduce these experiments in this study.

Source-context aware translation has also been attempted outside of the
predominant statistical machine translation framework. \cite{MBMT} implement a
simple form of example-based machine translation that is word-based and relies
chiefly on classifiers for the translation model component. Two studies derive
from the same concept while transcending a word-based paradigm:
\cite{MARKERBASED} use chunks delimited by common markers. and \cite{PBMBMT}
attempts a full extension to phrases similar to SMT. Although positive results
are achieved in the latter study, it does not rival state-of-the-art SMT.

The most important and complete study we build upon is \cite{Rejwanul+11},
which in turn draws from the majority of the aforementioned studies, and
provides an extensive comparison between them. Their study finds that including
such linguistically-informed contextual features in general produces
improvements.  The main contrast between our study and theirs is that they
focus on a variety of linguistically-informed contextual features, whereas we
depart from a purely unsupervised angle and intend to settle some of the
conflicting reports whether this may lead to an improvement in translation
quality. A notable focus in our study will be possible methods of integrating the
classifier probabilities in the SMT, as recommended also by \cite{Gimenez+07}.


\section{Methodology}

\subsection{Modelling source-side context with classifiers}

In line with several previous studies \cite{Rejwanul+11,PBMBMT,
Stroppa+07,MARKEDBASED}, we make use of memory-based classifiers to build a
translation model informed by source-side context information. More
specifically, we will be using IB1 \cite{IB1}, an implementation of k-Nearest
Neighbours; IGTree, an optimised and lossless tree-based approximation thereof;
and TRIBL2, a mixture of the two. These algorithms are implemented in the
software Timbl \cite{TIMBL}.\footnote{\url{http://ilk.uvt.nl/timbl}} These
algorithms are well-suited for symbolic data and highly multivariate classes.
Moreover, memory-based classification has been a proven solution in the field
of Word Sense Disambiguation \cite{SENSEVAL2,WSD2}



\subsection{Integration in an SMT Decoder}

The task of an SMT decoder is to find the best translation amongst
a vast pool of possible translation hypotheses. The best translation hypothesis
is the translation hypothesis that maximises a log linear equation. This
log-linear equation draws from various models, such as a translation model
(i.e. the phrase-translation table), a target-language model optionally
additional models such as a distortion model or word-reordering model.

The translation model is a mapping of the set of source phrases ($S$) to the
set of target phrases ($T$). Each phrase-pair $(s,t)$ where $s \elem S$ and $t
\elem T$ is in described by a score vector indicating the likelihood of
translation. This score vector most notably consists of the probababilities
$p(s|t)$ and $p(t|s)$. In addition, lexical weighting probabilities $lex(s|t)$
and $lex(t|s)$ express the probability of a phrase-pair word-by-word, and are
often included as components in the score vector. During decoding, the total
score of the translation model and other models is expressed as a log-linear
combination, in which different weights can be assigned to each of the
components of the score vector. These weights are parameters to the task and
are typically optimised automatically on development data using for instance
Minimum Error Rate Reduction \cite{MERT}.


For the modelling source-side context the component of the score
vector that 


The state-of-the art SMT decoder used in a majority of MT studies is Moses
\citep{MOSES}. However, it offers no facilities to take source-side context
information in account. We had to consider three options to achieve our goal of
integrating source-side context : 1) creating a new decoder; 2) enhancing
Moses; or 3) using a bypass method. We decided, in line with most of the
literature, to follow the third option, as it was the easiest and allows us to
use Moses as a black box.


immediately immediately draws our interest is $p(t|s)$. 

For the modelling source-side context the component of the score
vector that 

immediately immediately draws our interest is $p(t|s)$. 



Each of the components of the
score vector



The phrase-translation table embodies the translation model, and is 








Moses features, amongst others, a stack-based decoder. For each
decoding run, there are as many stacks as there are words in the source
sentence. Translation proceeds by translating 

charged with the tasked 

Integrating source-side context in 




\end{document}
