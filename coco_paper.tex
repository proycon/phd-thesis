%to be submitted to SoftwareX

\begin{document}


\section{Introduction}

The n-gram, a sequence of $n$ consecutive word tokens, is a core concept for
many Natural Language Processing (NLP) applications. One of the most NLP basic
tasks is to read corpus text and compute an $n$-gram frequency list, elementary
for any kind of statistical analysis. The unigram frequency list, i.e. word
frequency list, is the simplest instance of this task that is especially
ubiquitous. Computing $n$-gram frequency on a corpus text is fairly trivial,
and any beginning computer science student will have no trouble to accomplish
this in just a few lines of code in a modern high-level programming language.
Nevertheless, this computation is at the heart of Colibri Core, the NLP sofware
we aim to introduce in this paper, as there is more to it than meets the eye at
first.

N-grams are typically distributed in a Zipfian fashion, implying there are only
a few high-frequency patterns, with words such as common function words in the
lead, and there is a long tail of patterns that occur only very sparsely. This
basic fact makes counting a notoriously memory-hungry enterprise, as patterns
occurring below your minimum frequency treshold can not be discarded from
memory until the entire corpus has been processed. 

When working with large data sets and higher-order $n$-ngrams, this memory
problem becomes apparent quickly when trivial solutions are employed. Colibri
Core, on the other hand, offers tools and programming libraries that are
heavily optimised to 1) reduce memory usage, and 2) have high performance.

Colibri Core is developed in the scope of the ``Constructions as Linguistic
Bridges'' PhD study, which roughly focusses on the role of constructions, i.e.
patterns such as $n$-grams, in their context, in Machine Translation. Machine
Translation, as well as Language Modelling are two common fields where large
data sets are often encountered, as more data generally improves the quality of
the resulting translation model or language model. 

The task of finding $n$-grams is generalised in Colibri Core to the task of
finding \emph{patterns} or \emph{constructions} (we use the terms
interchangeably). Furthermore, once patterns are identified, resulting in a
\emph{pattern model}, Colibri Core can extract relations between the patterns.

\section{Patterns}

We distinguish three categories of patterns, and define them as follows:

\begin{enumerate}
    \item N-grams - A sequence of $n$ word tokens that are all consecutive.
    \item Skipgrams - A fixed-length sequence of $p$ word tokens and $q$ token placeholders/wildcards ($n=p+q$), the placeholders constitute gaps or skips and a skipgram can contain multiple of these. In turn, a gap can span one or more tokens. 
    \item Flexgrams - A sequence s, with one or more gaps of variable length, which implies the pattern by itself is of undefined length.
\end{enumerate}

Our definitions are defined narrowly and, with exception of $n-gram$ do not
necessarily correspond to the way the concepts are used in other studies. Some
may use the term skipgram to include what we call flexgram.  The term
``elastigram'' is sometimes also employed to refer to flexgrams. 
%TODO: cite something







for even when you find a
certain pattern only once, you have to retain it in memory until the very end of
your counting procedure, as you can never be sure 


Working with large data sets 


Similarly, a count of n-grams 

building block 
Natural Language Processing
N-grams form the building block of countless Natural Language Processing (NLP) 

Working with large dat

This paper presents Colibri Core, 





\end{document}
