\chapter{Conclusion}
\label{chap:conclusion}

In this dissertation, we posited the main hypothesis that inclusion of source-side context information, without
linguistically informed features, benefits translation quality. We have
assessed this question from various different angles. We set out to answer
the following inter-related questions:

\begin{enumerate}
\item Can we improve translation by considering source-side context information? 
\item What context features prove most effective?
\item Are linguistically uninformed features effective?
\item How can source-side classifiers be used in translation tasks?
\item What optimisation techniques can we employ to achieve better translation quality?
\end{enumerate}

Chapter~\ref{chap:clwsd} showed an application in cross-lingual word sense
disambiguation, where word expert classifiers succesfully tackled the problem
of translating a word in context. Our system emerged as the winning system in
the SemEval 2010 Cross Lingual Word Sense Disambiguation task, and achieved
either winning or high scores in the same task in 2012. This second time
around, we placed focus on hyperparameter optimisation of the classifier
parameters and automatic feature selection per classifier expert, which
supersedes the voting approach we used the first time. Automatic feature
selection indeed leads to a modest improvement, whereas classifier parameter
optimisation does not.

At this stage, we still experimented with some basic linguistically informed
features as well, i.e. part-of-speech tags and lemmas. Lemmas proved to be
beneficial indeed, whereas part-of-speech tags failed to make a positive
impact. Nevertheless, our study's explicit focus is to assess the efficacy of
the surface features in their pure form, i.e. not enriched with any linguistic
information. We employed multiple classifiers or word experts in which the
feature vector consists of a simple local context window. In the cross-lingual WSD task our linguistically-uninformed
classifiers easily surpass the non-context-informed baselines. So for this
task, we find evidence that corroborates our main hypothesis and can answer yes
to both our first and third question.

We extensively experimented with global context features at this stage but find
a discrepancy in results between the SemEval 2010 and 2012 runs. Subsequent
attempts in later chapters support the SemEval 2012 findings, in which global
context features do not lead to an improvement in translation quality.













