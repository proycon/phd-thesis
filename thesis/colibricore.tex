
%Copyright Notice
%Authors who publish with this journal agree to the following terms:

%Authors retain copyright and grant the journal right of first publication with the work simultaneously licensed under a Creative Commons Attribution License that allows others to share the work with an acknowledgement of the work's authorship and initial publication in this journal.

%Authors are able to enter into separate, additional contractual arrangements for the non-exclusive distribution of the journal's published version of the work (e.g., post it to an institutional repository or publish it in a book), with an acknowledgement of its initial publication in this journal.

%By submitting this paper you agree to the terms of this Copyright Notice, which will apply to this submission if and when it is published by this journal.




\title{Efficient n-gram, skipgram and flexgram modelling with Colibri Core}

\begin{document}

\chapter{Efficient n-gram, skipgram and flexgram modelling with Colibri Core}

In this chapter, we do not yet tackle the issue of translation at all, but
instead focus on how we can computationally extract patterns in an efficient
way that also allows for modelling their context. This is a necessary
prerequisite for the further stages of the research.  The chapter has a strong
software-oriented focus and format. It explicitly introduces the software \emph{Colibri
Core}, which is the key component in the wider software employed in the research. This
software, however, also finds application in a much broader context.

Counting n-grams lies at the core of any frequentist corpus analysis and is
often considered a trivial matter. Going beyond consecutive n-grams to patterns
such as skipgrams and flexgrams increases the demand for efficient solutions.
The need to operate on big corpus data does so even more.  Lossless compression
and non-trivial algorithms are needed to lower the memory demands, yet retain
good speed. Colibri Core is software for the efficient computation of n-grams,
skipgrams and flexgrams from corpus data. The resulting pattern models can be
analyzed and compared in various ways. 

\textsc{This chapter is based on: } Efficient n-gram, skipgram and flexgram
modelling with Colibri Core. Journal of Open Research Software. Volume X, issue
Y.

\section{Introduction}

The n-gram, a sequence of $n$ consecutive word tokens, is a core concept for
many Natural Language Processing (NLP) applications. One of the most basic NLP
tasks is to read corpus text and compute an $n$-gram frequency list, elementary
for any kind of statistical analysis. The unigram frequency list, i.e.\ the
word frequency list, is the simplest instance of this task which is especially
ubiquitous. Computing $n$-gram frequency on a corpus text is fairly trivial,
and any beginning computer science student will have no trouble to accomplish
this in just a few lines of code in a modern high-level programming language.
However, optimising this to reduce memory constraints, improve speed, and scale
to large data, is a more complex matter. Colibri Core, the NLP software we
introduce here, offers efficient algorithms to do this.

N-grams are typically distributed in a Zipfian fashion, implying there are only
a few high-frequency patterns, with words such as common function words in the
lead, and there is a long tail of patterns that occur only very sparsely. This
basic fact makes counting a notoriously memory-hungry enterprise, as patterns
occurring below a minimum frequency treshold can not be discarded from
memory until the entire corpus has been processed sequentially. 

When working with large data sets and higher-order $n$-ngrams, this memory
problem becomes apparent quickly when trivial solutions are employed. Colibri
Core, on the other hand, offers tools and programming libraries that are
heavily optimised to 1) reduce memory usage, and 2) offer high-speed performance.

The task of finding $n$-grams is generalised in Colibri Core to the task of
finding \emph{patterns} or \emph{constructions} (we use the terms
interchangeably). Furthermore, once patterns are identified, resulting in a
\emph{pattern model}, Colibri Core can extract relations between the patterns.

As the name Colibri Core suggests, the software is geared to provide
\emph{core} functionality for modelling patterns and exposes this functionality
as a programming library as well as through command line tools. It aims to
provide a solid foundation upon which more specialised software can be built,
such as software for language modelling. The software is aimed at
NLP software developers and researchers with a solid technical background. 


\section{Implementation and Architecture}

\subsection{Patterns}

We distinguish three categories of patterns, and define them as follows:

\begin{enumerate}
    \item N-grams -- A sequence of $n$ word tokens that are all consecutive.
        For example: ``\texttt{to be or not to be}''
    \item Skipgrams -- A fixed-length sequence of $p$ word tokens and $q$ token
        placeholders/wildcards with total length $n$ ($n=p+q$), the
        placeholders constitute gaps or skips and a skipgram can contain
        multiple of these. In turn, a gap can span one or more tokens. For
    example: ``\texttt{``to \_ or \_ \_ be}''
    \item Flexgrams -- A sequence with one or more gaps of variable length,
        which implies the pattern by itself is of undefined length. For example:
        ``\texttt{to * or * be}''
\end{enumerate}

Our definitions are defined narrowly and, with exception of $n$-grams do not
necessarily match up precisely to the way the concepts are used in other studies. Some
may use the term skipgram to include what we call flexgram, or use another term
such as ``elastigram'' to refer to flexgrams. 

Skipgrams are used in the field to obtain a higher level of generalisation than
can be offered by n-grams. They can, for instance, be found as features in
classification tasks \cite{DHONDT}, or as a means in language modelling to
decrease perplexity \cite{Guthrie06}.

Dealing with word tokens implies that the corpus data has to be in a
tokenised form. We start from the basis of plain-text corpus data, containing one
\emph{unit} per line; a unit can correspond to a sentence, paragraph, tweet
or whatever unit is deemed appropriate for the task at hand. Corpus data can
alternatively be provided in FoLiA XML format \cite{FOLIAPAPER} as well, although linguistic
annotation will be ignored.

Text data is typically stored as a string of characters. The characters
themselves draw their denotation from a character encoding. The storage of a
huge amount of strings is inefficient from a memory perspective, considering
the fact that words follow a Zipfian distribution. Colibri Core therefore works
on the basis of a lossless compression, in which each unique word token is
assigned a numeric type identifier, which we call a \emph{class}. This
effectively defines the \emph{vocabulary} of your data, which we call a
\emph{class encoding}. A pattern is not represented as an array of
characters, but as an array of these classes instead. Further lossless
compression is achieved by holding this array of classes in a dynamic-length
byte representation, in which low class values can be stored in fewer bytes
than high class values. Classes $0$ to $127$ can be stored in a single byte,
higher classes require at least two bytes. To achieve maximum compression,
classes are assigned to word tokens based on frequency (i.e.\ entropy
encoding): words with the highest frequency receive the lowest classes.  This
is essentially a variant of Huffman coding \citep{HUFFMAN}. Some of the lowest
classes are reserved for special purposes, e.g. to delimit sentences (class 0) or as
markers for out-of-vocabulary words (class 2) or skipped content (classes 3 and 4).

Of each byte in the class representation, the highest bit is reserved as a continuation
marker. As long as the continuation marker is set, the next byte is still part of
the class. When it is low, we know we are at the final byte of a class
representation. The class itself is stored in the remaining 7-bits of each
byte. In practise this results in good compression and reduces memory usage; an
example corpus taking up $221$ MB on disk in plaintext form is reduced to just $60$ MB
when compressed.

To encode a text corpus, a class encoding needs to be computed first. To decode
the encoded corpus back to plain text, the class encoding is needed again.
Colibri Core provides tools and exposes library functions to do this.

\subsection{Informed Iterative Counting}

N-gram frequency lists are often parametrised by a certain threshold. All
$n$-grams below this occurrence threshold are pruned. We can circumvent the
problem of having to hold a huge amount of patterns in memory that do not meet
the threshold, as is typical in a Zipfian distribution. We do this by employing
informed counting. Informed counting is an iterative procedure, shown in pseudo
code in Algorithm~\ref{alg:ngramcounting}. Here we take $m$ to be the maximum
$n$-gram order we intend to extract. The whole corpus is then processed for
each $n$ where $1<n\leq m$, extracting the respective $n$-grams at each
iteration. This means that at each iteration, we can consult the results of the
previous iteration. We can then readily discard any $n$-gram with $n>1$ for
which either of the two $n-1$-grams it contains does not
meet the threshold, as it follows that the $n$-gram can never meet the
threshold either. By outright discarding an $n$-gram we do not need to store it
and its count in memory. After each iteration of $n$, we prune all the
$n$-grams that did not reach the threshold.


\begin{algorithm} \caption{Informed Iterative Counting for n-grams.  Take $m$
to be the maximum $n$-gram order we intend to extract, $t$ to be the minimum occurrence threshold, and $M$ to be the
pattern model in memory, with unigrams already counted in the more trivial fashion.}
\label{alg:ngramcounting}
\begin{algorithmic}
\For {$n \in 2..m$}
    \For {$line \in corpus$}
        \For {$ngram \in extractngrams(line,n)$}
            \State  $nm1gram_1, nm1gram_2 \leftarrow extractngrams(ngram,n-1)$
            \If {$M(nm1gram_1) \geq t$ \& $M(nm1gram_2) \geq t$}
                \State $M(ngram) \leftarrow M(ngram) + 1$
            \EndIf
        \EndFor 
    \EndFor
    \State $M \leftarrow prunengrams(M,n,t)$
\EndFor \\
\Return{M}
\end{algorithmic}
\end{algorithm}

Though not expressed in the simplified algorithm above, the actual
implementation accounts for more parameters, such as setting a lower bound to
$n$. The amount of back-off, going all the way up to $m-1$ here, can also be
fine-tuned.

A performance evaluation of this algorithm will be addressed later in the
section on \emph{Quality Control}.

\subsection{Informed Skipgram Counting}
\label{sec:skipgramcount}

The computation of skipgrams is parametrised by an upper limit $l\leq m$ in the number of
tokens, as the possible configuration of gaps increases exponentially with the
total length spanned. It first requires a count of all $n$-grams where $0<n\geq l$. 

The algorithm, shown in Algorithm~\ref{alg:skipgramcount} considers all
possible ways skips can be inserted in all of the n-grams in the model. It can
discard a skipgram candidate by looking at the non-skipped parts that make up the
skipgram, and checking whether those exceed the set threshold. 

\begin{algorithm} \caption{Informed Counting for skipgrams.  Take $l$
to be the maximum skipgram order we intend to extract, $t$ to be the minimum occurrence threshold, and $M$ to be the
pattern model in memory, with ngrams already counted.}
\label{alg:skipgramcount}
\begin{algorithmic}
\For {$n \in 3..l$}
    \For {$ngram \in getngrams(M,n,t)$}
        \For {$skipgram \in possibleconfigurations(ngram)$}
            \State $docount \leftarrow True$
            \For {$part \in parts(skipgram)$}
            \If {$M(part) < t$}
                    \State $docount \leftarrow False$
                    Break
                \EndIf
            \EndFor 
            \If {$docount$}
                \State $M(skipgram) \leftarrow M(skipgram) + 1$
            \EndIf
            \EndFor 
            \EndFor
    \State $M \leftarrow pruneskipgrams(M,n,t)$
\EndFor \\
\Return{M}
\end{algorithmic}
\end{algorithm}

In this algorithm, the $possibleconfigurations(ngram)$ function returns
all possible skipgram configurations for the given $n$-gram. Note that
the configuration of gaps depends only on the length of the $n$-gram, regardless
of its content, and can therefore easily be pre-computed. The
$parts(skipgram)$ function returns all consecutive parts that are
subsumed in the skipgram, i.e.\ the parts delimited by the gaps.

Like Algorithm~\ref{alg:ngramcounting}, Algorithm~\ref{alg:skipgramcount}
assumes a threshold $t>1$. When $t=1$, more trivial algorithms are
invoked, as the user does not want to prune anything. These make only a single
pass over the data. For skipgrams this leads to an explosion of
resulting patterns, exponential with number of tokens, and is best avoided.

\subsection{What counts?}
\label{sec:whatcounts}

The counting algorithms are parametrised by a various other parameters which
are not shown in Algorithm~\ref{alg:ngramcounting} and
Algorithm~\ref{alg:skipgramcount} to reduce complexity. The wide variety of
parameters allow the user to influence precisely what is counted and this is one
of the main assets of Colibri Core. Parameters exist to affect the following:

\begin{itemize}
    \item The minimum and maximum length (in words/tokens) of the n-grams
        and/or skipgrams to be extracted. Setting minimum and maximum length to
        the same value will produce a model of homogenous pattern length
        (e.g.\ only trigrams or words).
    \item A secondary \emph{word} occurrence threshold can be configured. This is a value set higher than
        the primary occurrence threshold. Only patterns
        occuring above the primary threshold, and for which each of the
        individual words/unigrams in the pattern passes the secondary threshold as well, will
        be included in the model.
    \item N-grams that are not subsumed by higher order n-grams, i.e.\ which do
        not  occur as part of a higher order n-gram in the data/model, can be pruned
        from the model. This allows you to extract for instance all trigrams
        and all bigrams and unigrams that make up the trigrams, but not the
        bigrams and unigrams that are not subsumed in trigrams. 
    \item Skipgrams can be constrained using the \emph{skip type threshold}. This
        requires that at least this number of distinct patterns must fit in the
        gaps of the skipgram. Higher values will produce less skipgrams, but
        typically more generic ones. A skipgram such as \emph{The \_ house} will
        then only be included in the model if the corpus has instances in which the gap is
        filled by at least the specified number of distinct patterns. 
        An example is if the threshold is set to 2, and the corpus contains \emph{The
        big house} and \emph{The small house}. If the corpus  only has
        one of these instantiations, and no other instantiations either, then the skipgram would not be included.
    \item Skipgrams and n-grams are typically computed using the same
        occurrence threshold, but it is also possible to use a different threshold
        for skipgrams.
\end{itemize}

\subsection{Pattern Models}

A pattern model is a $key \mapsto value$ store, where the keys correspond to
patterns and the values typically correspond to occurrence counts, although any
kind of other value is supported too. Our aim with pattern models is to have a
data structure that allows for \emph{quick} lookup and iteration, as well as
quick insertion during training. Moreover, memory consumption should be as
conservative as possible, to allow handling of big data.

The underlying C++ library allows for choosing the actual underlying container
implementation and value type through \emph{templating}. The default container
datatype is a hash map\footnote{using the \texttt{unordered\_map} STL container
in C++11}, which guarantees $O(1)$ access and update times under ideal hashing
conditions. The hash\footnote{Spooky Hash v2 is used for hashing:
http://burtleburtle.net/bob/hash/spooky.html} is computed directly from the
binary representation of a pattern. Storing each pattern individually results
in a lot of redundant information to be stored, as patterns overlap to a large
degree. To conserve memory, the models can store pattern pointers\footnote{Each
pattern pointer takes up 16 bytes} instead. These point to the original corpus
data.

The use of hash maps can be contrasted to the use of suffix (or prefix) tries,
a common datastructure for storing n-grams. Although suffix tries also benefit
from decreased memory use due to no overlap in pattern data, the strongly
linked nature of tries causes a significant overhead in memory
use\footnote{Each pointer consuming 8 bytes on 64-bit architectures, and one
would be needed between every two tokens.} that quickly exceeds the memory
footprint of hash maps.  For this reason, hash maps are the default and tries
are currently not implemented in Colibi Core. The templating, however, does
allow for such an implementation to be added in the future.

At this point, we need to address suffix arrays \cite{Manber90} as well, which
are derived from suffix tries but with significantly decreased space
requirements. Suffix arrays with longest common prefix (LCP) arrays will
consume less memory than our hash maps, but are typically much slower to
construct and query. Though no exhaustive experiment was conducted to this
end, we did compare a predecessor of Colibri Core with a suffix array implementation
\cite{Stehouwer10} and found our implementation to be significantly faster in model
construction.

We distinguish two types of pattern model, depending on the type of the values,
which in the underlying C++ implementation is subject to templating as well:

\begin{enumerate}
 \item \textbf{Unindexed Pattern Models} -- Values are simple integers
 \item \textbf{Indexed Pattern Models} -- Values are arrays of indices where
     the pattern's occurrences in the corpus are stored.
\end{enumerate}

Obviously, indexed pattern models make a considerably higher demand on memory.
They do, however, allow for a broader range of computations, as shall become
apparent in subsequent sections.

\subsection{Two-step training}

Training indexed patterns models is more memory intensive than training
unindexed models, especially in very large corpora (say at least a hundred
million words). To lower the demand on memory for such corpora, we implement a
\emph{two-step training} procedure. This involves first constructing an
unindexed pattern model and subsequently constructing an indexed model on the
basis of that, by making another pass over the corpus and gathering all
indices. The gain here is in avoiding temporary storage of the indices that
will not pass the occurrence threshold but that cannot be ruled out a priori by the
informed counting algorithm.  This conservation of memory comes at the cost of
a extended execution time.

\subsection{Corpus Comparison}

The computation of pattern models on two or more distinct corpora, provided the
class encoding is the same for all of them, provides a basis for comparative
corpus analysis. One measure for corpus comparison introduced in the software
is the notion of \emph{coverage}. This metric is expressed as the number of
tokens in the test corpus that is covered by the patterns from the training
corpus, it is therefore asymmetric and the choice of training and test corpus
matters.  The metric can be represented either in absolute counts, or in
normalised form as a fraction of the total amount of tokens in the test corpus.

To perform such comparisons, we first compute a pattern model on the training
corpus, and subsequently compute a second pattern model on the test corpus, but
\emph{constrained} by the former pattern model. The ability to train
constrained models is present throughout the software and can for instance also
be used to train a pattern model based on a custom preset list of patterns,
effectively limiting the model to this preselection. The previously described
two-step training algorithm is also an example of constrained training.

Summarised statistics are computed at multiple levels. Measures such as
occurrence count and coverage can be consulted for aggregates of n-grams,
skipgrams, or flexgrams, as well as specific patterns. The former two can be
inspected specifically for each of the different pattern sizes present in the
model, i.e.\ for each value of $n$.

The coverage metric is a fairly crude metric of corpus overlap, despite the
ability to assess it for different aggregates. A more widely established metric
for corpus comparison is \emph{log-likelihood}. Log likelihood expresses how much more
likely any given pattern is for either of the two models. It therefore allows
you to identify how indicative a pattern is for a particular corpus. Our
implementation follows the methodology of~\cite{Rayson00comparingcorpora}.

\subsection{Relations between Patterns}

Various relations can be extracted between patterns in a pattern model, either
through the API or a dedicated query tool. For all but the first of the
relation types an indexed pattern model is required. 

\begin{itemize}
 \item \textbf{Subsumption Relation} -- $n$-gram $x y z$ subsumes $n-1$-grams $x y$ and $y z$. 
 \item \textbf{Succession Relation} -- Patterns that occur in a sequence in the
     corpus data. For example: pattern $x$ precedes $yz$ and pattern $z$ succeeds $xy$.
 \item \textbf{Instantiation Relation} -- Skipgrams or flexgrams may be
     \emph{instantiated} by other patterns. For example, ``to be \_ not \_'' be
     is instantiated by ``or \_ to'', resulting in the 6-gram ``to be or not to be''. This type of relation thus allows you to precisely determine what patterns occur in certain gaps.
 \item \textbf{Co-occurrence Relation} -- Different patterns can naturally co-occur multiple times
     within the the structural ``units'' you decided upon for the corpus (e.g.\ 
     sentences, paragraphs, tweets, etc). These units are newline delimited in
     your original input. The measure of such co-occurrence 
     can be expressed by established metrics such as Jaccard and (normalised) mutual
     pointwise information.
\end{itemize}

For each of these categories, the relationship is bidirectional, i.e.\ you can
query for the subsuming patterns as well as the subsumed patterns, the left
neighbours as well as the right neighbours, the instantiations as well as the
abstractions. The co-occurrence relationship is fully symmetrical. 

These latter three relationships rely on both the \emph{forward index} inherent
in an indexed pattern model, as well as the \emph{reverse index}, a function
from positions in the corpus to an array of patterns that are found at said
position. The reverse index is not modelled in memory as an explicit mapping from
positions to patterns, but computed on-the-fly based on the loaded corpus data
and a simple reverse index keeping track of where each unit/sentence starts.

\subsection{Flexgram Counting}

Thus far, we have explained the algorithms for n-gram counting and skipgram
counting, but have not yet done so for flexgrams, i.e. patterns with
variable-width gaps. The implementation allows flexgrams to be computed in two
different ways. The first is by extracting skipgrams first, and then
abstracting flexgrams from these skipgrams. In this case the flexgram
computation is constrained by the same maximum-size limit under which the
skipgrams have been extracted.  The second method for flexgram extraction
proceeds through the co-occurence relation. A flexgram is formed whenever two
patterns (within the same structural unit) occur above a set threshold. The
implementation of this latter method is currently limited to flexgrams with a
single variable-width gap. This method is recommended when the user is
interested in long-distance flexgrams, whereas the abstraction method is
recommended when the user is more interested in having multiple gaps or
the relationship between flexgrams and skipgrams.

\section{Quality Control}
\label{sec:qc}

\subsection{Unit tests}

To ensure the software is working as intended, an extensive series of unit
tests is available.  The tool \texttt{colibri-test} tests the various functions
of the C++ library. The \texttt{test.py} script tests the Python binding.
Testing in the form of continuous integration is made possible through
\emph{Travis CI}, where all test results are publicly available for inspection.
\footnote{\url{https://travis-ci.org/proycon/colibri-core}}

\subsection{Performance Evaluation}

A performance evaluation of our iterative counting algorithm has been performed
by comparing our optimised C++ implementation with a naive implementation in
Python, as well as a simple implementation of informed iterative counting in
Python. These baselines are unindexed and simply store occurrence count in a
Python dictionary. Comparisons can also be made between Colibri Core's
unindexed vs indexed models, and between the optimised pointer models vs
standard pattern models.  For the experiments, shown in
Table~\ref{tab:benchmarks} we used a corpus of Dutch
translations of TED talks of $127,806$ sentences and $2,330,075$
words\footnote{The data is from the IWSLT 2012 Evaluation Campaign,
    \url{http://hltc.cs.ust.hk/iwslt/index.php/evaluation-campaign/ted-task.html\#MTtrack}.
Tokenisation was performed using ucto, \url{http://ilk.uvt.nl/ucto}.}. We set
the occurrence threshold ($t$) to $2$ and extract everything from unigrams to
$8$-grams. 

\begin{table}[h]
\footnotesize{
\begin{tabular}{lll}
Experiment & CPU time & Memory  \\
\hline
Naive Python implementation & $37.2$ s & $2848$ MB \\
Python with iterative counting  & $65.7$ s & $170$ MB \\
\hline
Unindexed Pattern Model (from file) & $14.6$ s & $64$ MB ($76$ MB) \\
Unindexed Pattern Model (preloaded corpus) & $11.3$ s & $64$ MB ($76$ MB) \\
Unindexed Pattern Model (preloaded corpus), with skipgrams & $60.4$ s & $118$ MB ($131$ MB) \\
Unindexed Pattern Pointer Model (preloaded corpus)  & $9.4$ s & $50$ MB ($62$ MB) \\
Indexed Pattern Model (preloaded corpus) & $13.6$ s & $148$ MB ($160$ MB) \\
Indexed Pattern Model (preloaded corpus), with skipgrams & $24.2$ s & $165$ MB ($178$ MB) \\
Indexed Pattern Pointer Model (preloaded corpus) & $11.0$ s & $134$ MB ($146$ MB) \\
Unindexed Pattern Model (ordered map) & $30.5$ s & $84$ MB ($96$ MB) \\
\end{tabular}
\caption{Colibri Core performance benchmarks using the \texttt{colibri-benchmarks} tool and the 
\texttt{benchmarks.py} script for the Python baselines. Memory usage is measured as the difference in resident memory after training
and before training. Peak memory usage is measured absolutely as reported by
the OS and included in parentheses. All experiments were performed on a Linux system
with an Intel Xeon CPU (E5-2630L v3) at 1.80GHz and 256GB RAM.
}
}
\label{tab:benchmarks}
\end{table}

From the data in Table~\ref{tab:benchmarks} we observe that the Python
implementation of iterative counting already shows a drastic reduction in
memory usage compared to the naive python implementation\footnote{The naive
    implementation also seems to not recover well from peak memory consumption
after pruning of all patterns occurring only once.}. The first two experiments
with Colibri Core's unindexed pattern models shows that memory is succesfully
reduced further due to the class encoding. It also demonstrates the clear
advantage in execution time offered by Colibri Core.

The pattern pointer models prove capable of further reduction in memory
consumption, and offer a clear speed advantage. The pointer models use a
representation of patterns that refer to the original corpus data, which is
fully loaded into memory, rather than storing a separate copy for each pattern. 

The C++ library allows for easy swapping of the underlying datastructure for
pattern models. The default hashmap (\texttt{std::unordered\_map}) solution can
be contrasted to a pattern model using using an ordered map
(\texttt{std::map}\footnote{the hash key is still computed as it determines
the ordering}, the last experiment in the table). As expected, the ordered map proves to be significantly slower
and does not yield a memory advantage either.

\subsection{Documentation} 

Extensive documentation, including API references for both Python and C++, is provided at
\url{https://proycon.github.io/colibri-core}. An interactive tutorial for Python is also
available.

\section{Technical Availability}

\subsection{Operating System}

Colibri Core should be able to run on modern POSIX-compliant operating systems, including
Linux, FreeBSD and Mac OS X. It is tested to compile with current versions of
both \texttt{gcc} as well as \texttt{clang}.

\subsection{Programming Language}

Colibri Core is written in C++, adhering to the C++11 standard. The Python
binding is written in Cython (0.23 or above) and supports both Python 2.7 as
well as Python 3. The latter is recommended.

The Python binding ensures that the functionality of Colibri Core is easily
accessible from Python without sacrificing the great performance benefit native
code provides. Python was chosen as it is a high-level programming language in
widespread use in the scientific community, and the NLP community in
particular. It demands less expertise from the developer than C++ and is more
suitable for rapid prototyping.

\subsection{Additional system requirements}

Colibri Core provides memory-based techniques where models are held entirely in
memory to guarantee maximum performance on lookup and computation.  This
approach can be contrasted to e.g.\ database approaches which have much higher
latency.  It does place considerable memory requirements on the system on which
is it run, though this depends entirely on the size of the data and the
thresholds the user uses. We recommend at least 16GB RAM. In practise, using Colibri
Core on high-end computing servers with 256GB RAM is not uncommon for extensive
computation on big data sets.

Colibri Core is single-threaded due to the non-distributable nature of most of the
algorithms. A 64-bit architecture is required, 32-bit is not supported.

\subsection{Dependencies}

Colibri Core relies on the standard C/C++ library and a full build environment including autoconf and
automake; Python 2.7 or 3; Cython 0.23 or above. Support for reading the
\emph{FoLiA} XML format for text is entirely optional and requires the \texttt{libfolia}
library.\footnote{https://proycon.github.io/folia}


\subsection{Software Location}

\subsubsection*{Archive}

\begin{addmargin}[2em]{2em}
\textbf{Name}: Zenodo \\
\textbf{Persistent Identifier}: \url{https://dx.doi.org/10.5281/zenodo.33464} \\
\textbf{Publisher}: Maarten van Gompel \\
\textbf{Licence}: GNU General Public Licence v3 \\
\textbf{Date published}: November 8th, 2015 \\
\end{addmargin}

\subsubsection*{Code repository}

\begin{addmargin}[2em]{2em}
\textbf{Name}: GitHub \\
\textbf{Identifier}: \url{https://github.com/proycon/colibri-core} \\
\textbf{Website}: \url{https://proycon.github.io/colibri-core} \\
\textbf{Licence}: GNU General Public Licence v3 \\
\textbf{Date published}: since September 21st, 2013 \\
\end{addmargin}


\section{Reuse potential}

Colibri Core explicitly aims to provide a foundation for researchers in the NLP
community to build their tools and research on. The software is already being
employed in ongoing research on Machine
Translation\footnote{\url{https://github.com/proycon/colibri-mt}}, Bayesian Language
Modelling\footnote{\url{https://github.com/naiaden/cococpyp}}, Kneser-Ney Language
Modelling\footnote{\url{https://github.com/naiaden/apodiformes}}, spelling
correction\footnote{\url{https://github.com/proycon/gecco}}, and event
prediction in social media streams\footnote{\url{https://github.com/fkunneman/ADNEXT\_predict}}.

As a programming library for both C++ and Python, Colibri Core can be
potentially adopted by a wide variety of third party developers. As a set of
tools and scripts, Colibri Core also has merit standalone. It is, however,
focussed on command-line usage and therefore still requires a certain technical
expertise from the end-user.

To increase the accessibility of Colibri Core, a RESTful webservice as well as
generic web interface is already provided through CLAM\cite{CLAMPAPER}. With this we
hope to meet the needs of less technical end-users, as well as automated
networked clients. This webservice is hosted at
\url{https://webservices-lst.science.ru.nl}.

Future work building upon Colibri Core may focus on offering 
high-level user-interfaces to reach a wider audience.



\end{document}
