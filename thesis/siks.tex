%Obtained from http://www.siks.nl/siks_dissertations.ltx at 2020-01-08
\chapter*{SIKS Dissertation Series}
\addcontentsline{toc}{chapter}{SIKS Dissertation Series}
\markboth{SIKS Dissertation Series}{SIKS Dissertation Series}

%\begin{longtable}{@{}l@{ }l@{\hspace{1em}}X}
\begin{longtabu}{@{}l@{ }l@{\hspace{1em}}X}
\toprule
2011	&	 01	&	 Botond Cseke (RUN), Variational Algorithms for Bayesian Inference in Latent Gaussian Models\\
	&	 02	&	 Nick Tinnemeier (UU), Organizing Agent Organizations. Syntax and Operational Semantics of an Organization-Oriented Programming Language\\
	&	 03	&	 Jan Martijn van der Werf (TUE), Compositional Design and Verification of Component-Based Information Systems\\
	&	 04	&	 Hado van Hasselt (UU), Insights in Reinforcement Learning; Formal analysis and empirical evaluation of temporal-difference\\
	&	 05	&	 Bas van der Raadt (VU), Enterprise Architecture Coming of Age - Increasing the Performance of an Emerging Discipline.\\
	&	 06	&	 Yiwen Wang (TUE), Semantically-Enhanced Recommendations in Cultural Heritage\\
	&	 07	&	 Yujia Cao (UT), Multimodal Information Presentation for High Load Human Computer Interaction\\
	&	 08	&	 Nieske Vergunst (UU), BDI-based Generation of Robust Task-Oriented Dialogues\\
	&	 09	&	 Tim de Jong (OU), Contextualised Mobile Media for Learning\\
	&	 10	&	 Bart Bogaert (UvT), Cloud Content Contention\\
	&	 11	&	 Dhaval Vyas (UT), Designing for Awareness: An Experience-focused HCI Perspective\\
	&	 12	&	 Carmen Bratosin (TUE), Grid Architecture for Distributed Process Mining\\
	&	 13	&	 Xiaoyu Mao (UvT), Airport under Control. Multiagent Scheduling for Airport Ground Handling\\
	&	 14	&	 Milan Lovric (EUR), Behavioral Finance and Agent-Based Artificial Markets\\
	&	 15	&	 Marijn Koolen (UvA), The Meaning of Structure: the Value of Link Evidence for Information Retrieval\\
	&	 16	&	 Maarten Schadd (UM), Selective Search in Games of Different Complexity\\
	&	 17	&	 Jiyin He (UVA), Exploring Topic Structure: Coherence, Diversity and Relatedness\\
	&	 18	&	 Mark Ponsen (UM), Strategic Decision-Making in complex games\\
	&	 19	&	 Ellen Rusman (OU), The Mind's Eye on Personal Profiles\\
	&	 20	&	 Qing Gu (VU), Guiding service-oriented software engineering - A view-based approach\\
	&	 21	&	 Linda Terlouw (TUD), Modularization and Specification of Service-Oriented Systems\\
	&	 22	&	 Junte Zhang (UVA), System Evaluation of Archival Description and Access\\
	&	 23	&	 Wouter Weerkamp (UVA), Finding People and their Utterances in Social Media\\
	&	 24	&	 Herwin van Welbergen (UT), Behavior Generation for Interpersonal Coordination with Virtual Humans On Specifying, Scheduling and Realizing Multimodal Virtual Human Behavior\\
	&	 25	&	 Syed Waqar ul Qounain Jaffry (VU), Analysis and Validation of Models for Trust Dynamics\\
	&	 26	&	 Matthijs Aart Pontier (VU), Virtual Agents for Human Communication - Emotion Regulation and Involvement-Distance Trade-Offs in Embodied Conversational Agents and Robots\\
	&	 27	&	 Aniel Bhulai (VU), Dynamic website optimization through autonomous management of design patterns\\
	&	 28	&	 Rianne Kaptein (UVA), Effective Focused Retrieval by Exploiting Query Context and Document Structure\\
	&	 29	&	 Faisal Kamiran (TUE), Discrimination-aware Classification\\
	&	 30	&	 Egon van den Broek (UT), Affective Signal Processing (ASP): Unraveling the mystery of emotions\\
	&	 31	&	 Ludo Waltman (EUR), Computational and Game-Theoretic Approaches for Modeling Bounded Rationality\\
	&	 32	&	 Nees-Jan van Eck (EUR), Methodological Advances in Bibliometric Mapping of Science\\
	&	 33	&	 Tom van der Weide (UU), Arguing to Motivate Decisions\\
	&	 34	&	 Paolo Turrini (UU), Strategic Reasoning in Interdependence: Logical and Game-theoretical Investigations\\
	&	 35	&	 Maaike Harbers (UU), Explaining Agent Behavior in Virtual Training\\
	&	 36	&	 Erik van der Spek (UU), Experiments in serious game design: a cognitive approach\\
	&	 37	&	 Adriana Burlutiu (RUN), Machine Learning for Pairwise Data, Applications for Preference Learning and Supervised Network Inference\\
	&	 38	&	 Nyree Lemmens (UM), Bee-inspired Distributed Optimization\\
	&	 39	&	 Joost Westra (UU), Organizing Adaptation using Agents in Serious Games\\
	&	 40	&	 Viktor Clerc (VU), Architectural Knowledge Management in Global Software Development\\
	&	 41	&	 Luan Ibraimi (UT), Cryptographically Enforced Distributed Data Access Control\\
	&	 42	&	 Michal Sindlar (UU), Explaining Behavior through Mental State Attribution\\
	&	 43	&	 Henk van der Schuur (UU), Process Improvement through Software Operation Knowledge\\
	&	 44	&	 Boris Reuderink (UT), Robust Brain-Computer Interfaces\\
	&	 45	&	 Herman Stehouwer (UvT), Statistical Language Models for Alternative Sequence Selection\\
	&	 46	&	 Beibei Hu (TUD), Towards Contextualized Information Delivery: A Rule-based Architecture for the Domain of Mobile Police Work\\
	&	 47	&	 Azizi Bin Ab Aziz (VU), Exploring Computational Models for Intelligent Support of Persons with Depression\\
	&	 48	&	 Mark Ter Maat (UT), Response Selection and Turn-taking for a Sensitive Artificial Listening Agent\\
	&	 49	&	 Andreea Niculescu (UT), Conversational interfaces for task-oriented spoken dialogues: design aspects influencing interaction quality\\

\midrule
2012&	 01	&	 Terry Kakeeto (UvT), Relationship Marketing for SMEs in Uganda\\
	&	 02	&	 Muhammad Umair (VU), Adaptivity, emotion, and Rationality in Human and Ambient Agent Models\\
	&	 03	&	 Adam Vanya (VU), Supporting Architecture Evolution by Mining Software Repositories\\
	&	 04	&	 Jurriaan Souer (UU), Development of Content Management System-based Web Applications\\
	&	 05	&	 Marijn Plomp (UU), Maturing Interorganisational Information Systems\\
	&	 06	&	 Wolfgang Reinhardt (OU), Awareness Support for Knowledge Workers in Research Networks\\
	&	 07	&	 Rianne van Lambalgen (VU), When the Going Gets Tough: Exploring Agent-based Models of Human Performance under Demanding Conditions\\
	&	 08	&	 Gerben de Vries (UVA), Kernel Methods for Vessel Trajectories\\
	&	 09	&	 Ricardo Neisse (UT), Trust and Privacy Management Support for Context-Aware Service Platforms\\
	&	 10	&	 David Smits (TUE), Towards a Generic Distributed Adaptive Hypermedia Environment\\
	&	 11	&	 J.C.B. Rantham Prabhakara (TUE), Process Mining in the Large: Preprocessing, Discovery, and Diagnostics\\
	&	 12	&	 Kees van der Sluijs (TUE), Model Driven Design and Data Integration in Semantic Web Information Systems\\
	&	 13	&	 Suleman Shahid (UvT), Fun and Face: Exploring non-verbal expressions of emotion during playful interactions\\
	&	 14	&	 Evgeny Knutov (TUE), Generic Adaptation Framework for Unifying Adaptive Web-based Systems\\
	&	 15	&	 Natalie van der Wal (VU), Social Agents. Agent-Based Modelling of Integrated Internal and Social Dynamics of Cognitive and Affective Processes.\\
	&	 16	&	 Fiemke Both (VU), Helping people by understanding them - Ambient Agents supporting task execution and depression treatment\\
	&	 17	&	 Amal Elgammal (UvT), Towards a Comprehensive Framework for Business Process Compliance\\
	&	 18	&	 Eltjo Poort (VU), 	Improving Solution Architecting Practices\\
	&	 19	&	 Helen Schonenberg (TUE), What's Next? Operational Support for Business Process Execution\\
	&	 20	&	 Ali Bahramisharif (RUN), Covert Visual Spatial Attention, a Robust Paradigm for Brain-Computer Interfacing\\
	&	 21	&	 Roberto Cornacchia (TUD), Querying Sparse Matrices for Information Retrieval\\
	&	 22	&	 Thijs Vis (UvT), Intelligence, politie en veiligheidsdienst: verenigbare grootheden?\\
	&	 23	&	 Christian Muehl (UT), Toward Affective Brain-Computer Interfaces: Exploring the Neurophysiology of Affect during Human Media Interaction\\
	&	 24	&	 Laurens van der Werff (UT), Evaluation of Noisy Transcripts for Spoken Document Retrieval\\
	&	 25	&	 Silja Eckartz (UT), Managing the Business Case Development in Inter-Organizational IT Projects: A Methodology and its Application\\
	&	 26	&	 Emile de Maat (UVA), Making Sense of Legal Text\\
	&	 27	&	 Hayrettin Gurkok (UT), Mind the Sheep! User Experience Evaluation \& Brain-Computer Interface Games\\
	&	 28	&	 Nancy Pascall (UvT), Engendering Technology Empowering Women\\
	&	 29	&	 Almer Tigelaar (UT), Peer-to-Peer Information Retrieval\\
	&	 30	&	 Alina Pommeranz (TUD), Designing Human-Centered Systems for Reflective Decision Making\\
	&	 31	&	 Emily Bagarukayo (RUN), A Learning by Construction Approach for Higher Order Cognitive Skills Improvement, Building Capacity and Infrastructure\\
	&	 32	&	 Wietske Visser (TUD), 	Qualitative multi-criteria preference representation and reasoning\\
	&	 33	&	 Rory Sie (OUN), Coalitions in Cooperation Networks (COCOON)\\
	&	 34	&	 Pavol Jancura (RUN), Evolutionary analysis in PPI networks and applications\\
	&	 35	&	 Evert Haasdijk (VU), Never Too Old To Learn -- On-line Evolution of Controllers in Swarm- and Modular Robotics\\
	&	 36	&	 Denis Ssebugwawo (RUN), Analysis and Evaluation of Collaborative Modeling Processes\\
	&	 37	&	 Agnes Nakakawa (RUN), A Collaboration Process for Enterprise Architecture Creation\\
	&	 38	&	 Selmar Smit (VU), Parameter Tuning and Scientific Testing in Evolutionary Algorithms\\
	&	 39	&	 Hassan Fatemi (UT), Risk-aware design of value and coordination networks\\
	&	 40	&	 Agus Gunawan (UvT), Information Access for SMEs in Indonesia\\
	&	 41	&	 Sebastian Kelle (OU), Game Design Patterns for Learning\\
	&	 42	&	 Dominique Verpoorten (OU), Reflection Amplifiers in self-regulated Learning\\
	&	 43	&	 Withdrawn \\
	&	 44	&	 Anna Tordai (VU), On Combining Alignment Techniques\\
	&	 45	&	 Benedikt Kratz (UvT), A Model and Language for Business-aware Transactions\\
	&	 46	&	 Simon Carter (UVA), Exploration and Exploitation of Multilingual Data for Statistical Machine Translation\\
	&	 47	&	 Manos Tsagkias (UVA), Mining Social Media: Tracking Content and Predicting Behavior\\
	&	 48	&	 Jorn Bakker (TUE), Handling Abrupt Changes in Evolving Time-series Data\\
	&	 49	&	 Michael Kaisers (UM), Learning against Learning - Evolutionary dynamics of reinforcement learning algorithms in strategic interactions\\
	&	 50	&	 Steven van Kervel (TUD), Ontologogy driven Enterprise Information Systems Engineering\\
	&	 51	&	 Jeroen de Jong (TUD), Heuristics in Dynamic Sceduling; a practical framework with a case study in elevator dispatching\\

\midrule
2013&    01	&    Viorel Milea (EUR), News Analytics for Financial Decision Support\\
	&	 02	&	 Erietta Liarou (CWI), MonetDB/DataCell: Leveraging the Column-store Database Technology for Efficient and Scalable Stream Processing\\
	&	 03	&	 Szymon Klarman (VU), Reasoning with Contexts in Description Logics\\
	&	 04	&	 Chetan Yadati (TUD), Coordinating autonomous planning and scheduling\\
	&	 05	&	 Dulce Pumareja (UT), Groupware Requirements Evolutions Patterns\\
	&	 06	&	 Romulo Goncalves (CWI), The Data Cyclotron: Juggling Data and Queries for a Data Warehouse Audience\\
	&	 07	&	 Giel van Lankveld (UvT), Quantifying Individual Player Differences\\
	&	 08	&	 Robbert-Jan Merk (VU), Making enemies: cognitive modeling for opponent agents in fighter pilot simulators\\
	&	 09	&	 Fabio Gori (RUN), Metagenomic Data Analysis: Computational Methods and Applications\\
	&	 10	&	 Jeewanie Jayasinghe Arachchige (UvT), A Unified Modeling Framework for Service Design.\\
	&	 11	&	 Evangelos Pournaras (TUD), Multi-level Reconfigurable Self-organization in Overlay Services\\
	&	 12	&	 Marian Razavian (VU), Knowledge-driven Migration to Services\\
	&	 13	&	 Mohammad Safiri (UT), Service Tailoring: User-centric creation of integrated IT-based homecare services to support independent living of elderly\\
	&	 14	&	 Jafar Tanha (UVA), Ensemble Approaches to Semi-Supervised Learning Learning\\
	&	 15	&	 Daniel Hennes (UM), Multiagent Learning - Dynamic Games and Applications\\
	&	 16	&	 Eric Kok (UU), Exploring the practical benefits of argumentation in multi-agent deliberation\\
	&	 17	&	 Koen Kok (VU), The PowerMatcher: Smart Coordination for the Smart Electricity Grid\\
	&	 18	&	 Jeroen Janssens (UvT), Outlier Selection and One-Class Classification\\
	&	 19	&	 Renze Steenhuizen (TUD), Coordinated Multi-Agent Planning and Scheduling\\
	&	 20	&	 Katja Hofmann (UvA), Fast and Reliable Online Learning to Rank for Information Retrieval\\
	&	 21	&	 Sander Wubben (UvT), Text-to-text generation by monolingual machine translation\\
	&	 22	&	 Tom Claassen (RUN), Causal Discovery and Logic\\
	&	 23	&	 Patricio de Alencar Silva (UvT), Value Activity Monitoring\\
	&	 24	&	 Haitham Bou Ammar (UM), 	Automated Transfer in Reinforcement Learning\\
	&	 25	&	 Agnieszka Anna Latoszek-Berendsen (UM), 	Intention-based Decision Support. A new way of representing and implementing clinical guidelines in a Decision Support System\\
	&	 26	&	 Alireza Zarghami (UT), 	Architectural Support for Dynamic Homecare Service Provisioning\\
	&	 27	&	 Mohammad Huq (UT), 	Inference-based Framework Managing Data Provenance\\
	&	 28	&	 Frans van der Sluis (UT), 	When Complexity becomes Interesting: An Inquiry into the Information eXperience\\
	&	 29	&	 Iwan de Kok (UT), 	Listening Heads\\
	&	 30	&	 Joyce Nakatumba (TUE), 	Resource-Aware Business Process Management: Analysis and Support\\
	&	 31	&	 Dinh Khoa Nguyen (UvT), 	Blueprint Model and Language for Engineering Cloud Applications\\
	&	 32	&	 Kamakshi Rajagopal (OUN), 	Networking For Learning; The role of Networking in a Lifelong Learner's Professional Development\\
	&	 33	&	 Qi Gao (TUD), User Modeling and Personalization in the Microblogging Sphere\\
	&	 34	&	 Kien Tjin-Kam-Jet (UT), 	Distributed Deep Web Search\\
	&	 35	&	 Abdallah El Ali (UvA), Minimal Mobile Human Computer Interaction\\
	&	 36	&	 Than Lam Hoang (TUe), 	Pattern Mining in Data Streams\\
	&	 37	&	 Dirk B\"{o}rner (OUN), Ambient Learning Displays\\
	&	 38	&	 Eelco den Heijer (VU), 	Autonomous Evolutionary Art\\
	&	 39	&	 Joop de Jong (TUD), A Method for Enterprise Ontology based Design of Enterprise Information Systems\\
	&	 40	&	 Pim Nijssen (UM), Monte-Carlo Tree Search for Multi-Player Games\\
	&	 41	&	 Jochem Liem (UVA), Supporting the Conceptual Modelling of Dynamic Systems: A Knowledge Engineering Perspective on Qualitative Reasoning\\
	&	 42	&	 L\'{e}on Planken (TUD), Algorithms for Simple Temporal Reasoning\\
	&	 43	&	 Marc Bron (UVA), Exploration and Contextualization through Interaction and Concepts\\

\midrule
2014&	 01	&	 Nicola Barile (UU), Studies in Learning Monotone Models from Data\\
	&	 02	&	 Fiona Tuliyano (RUN), Combining System Dynamics with a Domain Modeling Method\\
	&	 03	&	 Sergio Raul Duarte Torres (UT), Information Retrieval for Children: Search Behavior and Solutions\\
	&	 04	&	 Hanna Jochmann-Mannak (UT), Websites for children: search strategies and interface design - Three studies on children's search performance and evaluation\\
	&	 05	&	 Jurriaan van Reijsen (UU), Knowledge Perspectives on Advancing Dynamic Capability\\
	&	 06	&	 Damian Tamburri (VU), Supporting Networked Software Development\\
	&	 07	&	 Arya Adriansyah (TUE), Aligning Observed and Modeled Behavior\\
	&	 08	&	 Samur Araujo (TUD), Data Integration over Distributed and Heterogeneous Data Endpoints\\
	&	 09	&	 Philip Jackson (UvT), Toward Human-Level Artificial Intelligence: Representation and Computation of Meaning in Natural Language\\
	&	 10	&	 Ivan Salvador Razo Zapata (VU), Service Value Networks\\
	&	 11	&	 Janneke van der Zwaan (TUD), An Empathic Virtual Buddy for Social Support\\
	&	 12	&	 Willem van Willigen (VU), Look Ma, No Hands: Aspects of Autonomous Vehicle Control\\
	&	 13	&	 Arlette van Wissen (VU), Agent-Based Support for Behavior Change: Models and Applications in Health and Safety Domains\\
	&	 14	&	 Yangyang Shi (TUD), Language Models With Meta-information\\
	&	 15	&	 Natalya Mogles (VU), Agent-Based Analysis and Support of Human Functioning in Complex Socio-Technical Systems: Applications in Safety and Healthcare\\
	&	 16	&	 Krystyna Milian (VU), Supporting trial recruitment and design by automatically interpreting eligibility criteria\\
	&	 17	&	 Kathrin Dentler (VU), Computing healthcare quality indicators automatically: Secondary Use of Patient Data and Semantic Interoperability\\
	&	 18	&	 Mattijs Ghijsen (UVA), Methods and Models for the Design and Study of Dynamic Agent Organizations\\
	&	 19	&	 Vinicius Ramos (TUE), 	Adaptive Hypermedia Courses: Qualitative and Quantitative Evaluation and Tool Support\\
	&	 20	&	 Mena Habib (UT), Named Entity Extraction and Disambiguation for Informal Text: The Missing Link\\
	&	 21	&	 Kassidy Clark (TUD), 	Negotiation and Monitoring in Open Environments\\
	&	 22	&	 Marieke Peeters (UU), Personalized Educational Games - Developing agent-supported scenario-based training\\
	&	 23	&	 Eleftherios Sidirourgos (UvA/CWI), 	Space Efficient Indexes for the Big Data Era\\
	&	 24	&	 Davide Ceolin (VU), Trusting Semi-structured Web Data\\
	&	 25	&	 Martijn Lappenschaar (RUN), 	New network models for the analysis of disease interaction\\
	&	 26	&	 Tim Baarslag (TUD), What to Bid and When to Stop\\
	&	 27	&	 Rui Jorge Almeida (EUR), 	Conditional Density Models Integrating Fuzzy and Probabilistic Representations of Uncertainty\\
	&	 28	&	 Anna Chmielowiec (VU), 	Decentralized k-Clique Matching\\
	&	 29	&	 Jaap Kabbedijk (UU), 	Variability in Multi-Tenant Enterprise Software\\
	&	 30	&	 Peter de Cock (UvT), 	Anticipating Criminal Behaviour\\
	&	 31	&	 Leo van Moergestel (UU), 	Agent Technology in Agile Multiparallel Manufacturing and Product Support\\
	&	 32	&	 Naser Ayat (UvA), 	On Entity Resolution in Probabilistic Data\\
	&	 33	&	 Tesfa Tegegne (RUN), Service Discovery in eHealth\\
	&	 34	&	 Christina Manteli (VU), 	The Effect of Governance in Global Software Development: Analyzing Transactive Memory Systems.\\
	&	 35	&	 Joost van Ooijen (UU), 	Cognitive Agents in Virtual Worlds: A Middleware Design Approach\\
	&	 36	&	 Joos Buijs (TUE), 	Flexible Evolutionary Algorithms for Mining Structured Process Models\\
	&	 37	&	 Maral Dadvar (UT), 	Experts and Machines United Against Cyberbullying\\
	&	 38	&	 Danny Plass-Oude Bos (UT), 	Making brain-computer interfaces better: improving usability through post-processing.\\
	&	 39	&	 Jasmina Maric (UvT), 	Web Communities, Immigration, and Social Capital\\
	&	 40	&	 Walter Omona (RUN), 	A Framework for Knowledge Management Using ICT in Higher Education\\
	&	 41	&	 Frederic Hogenboom (EUR), 	Automated Detection of Financial Events in News Text\\
	&	 42	&	 Carsten Eijckhof (CWI/TUD), 	Contextual Multidimensional Relevance Models\\
	&	 43	&	 Kevin Vlaanderen (UU), 	Supporting Process Improvement using Method Increments\\
	&	 44	&	 Paulien Meesters (UvT), 	Intelligent Blauw. Met als ondertitel: Intelligence-gestuurde politiezorg in gebiedsgebonden eenheden.\\
	&	 45	&	 Birgit Schmitz (OUN), 	Mobile Games for Learning: A Pattern-Based Approach\\
	&	 46	&	 Ke Tao (TUD), 	Social Web Data Analytics: Relevance, Redundancy, Diversity\\
	&	 47	&	 Shangsong Liang (UVA), 	Fusion and Diversification in Information Retrieval\\

\midrule
2015&	 01	&	 Niels Netten (UvA), Machine Learning for Relevance of Information in Crisis Response\\
	&	 02	&	 Faiza Bukhsh (UvT), Smart auditing: Innovative Compliance Checking in Customs Controls\\
	&	 03	&	 Twan van Laarhoven (RUN), Machine learning for network data\\
	&	 04	&	 Howard Spoelstra (OUN), Collaborations in Open Learning Environments\\
	&	 05	&	 Christoph B\"{o}sch (UT), Cryptographically Enforced Search Pattern Hiding\\
	&	 06	&	 Farideh Heidari (TUD), Business Process Quality Computation - Computing Non-Functional Requirements to Improve Business Processes\\
	&	 07	&	 Maria-Hendrike Peetz (UvA), Time-Aware Online Reputation Analysis\\
	&	 08	&	 Jie Jiang (TUD), 	Organizational Compliance: An agent-based model for designing and evaluating organizational interactions\\
	&	 09	&	 Randy Klaassen (UT), HCI Perspectives on Behavior Change Support Systems\\
	&	 10	&	 Henry Hermans (OUN), OpenU: design of an integrated system to support lifelong learning\\
	&	 11	&	 Yongming Luo (TUE), Designing algorithms for big graph datasets: A study of computing bisimulation and joins\\
	&	 12	&	 Julie M. Birkholz (VU), Modi Operandi of Social Network Dynamics: The Effect of Context on Scientific Collaboration Networks\\
	&	 13	&	 Giuseppe Procaccianti (VU), Energy-Efficient Software\\
	&	 14	&	 Bart van Straalen (UT), A cognitive approach to modeling bad news conversations\\
	&	 15	&	 Klaas Andries de Graaf (VU), Ontology-based Software Architecture Documentation\\
	&	 16	&	 Changyun Wei (UT), Cognitive Coordination for Cooperative Multi-Robot Teamwork\\
	&	 17	&	 Andr\'{e} van Cleeff (UT), Physical and Digital Security Mechanisms: Properties, Combinations and Trade-offs\\
	&	 18	&	 Holger Pirk (CWI), Waste Not, Want Not! - Managing Relational Data in Asymmetric Memories\\
	&	 19	&	 Bernardo Tabuenca (OUN), Ubiquitous Technology for Lifelong Learners\\
	&	 20	&	 Lois Vanh\'{e}e (UU), 	Using Culture and Values to Support Flexible Coordination\\
	&	 21	&	 Sibren Fetter (OUN), Using Peer-Support to Expand and Stabilize Online Learning\\
	&	 22	&	 Zhemin Zhu (UT), 	Co-occurrence Rate Networks\\
	&	 23	&	 Luit Gazendam (VU), Cataloguer Support in Cultural Heritage\\
	&	 24	&	 Richard Berendsen (UVA), 	Finding People, Papers, and Posts: Vertical Search Algorithms and Evaluation\\
	&	 25	&	 Steven Woudenberg (UU), Bayesian Tools for Early Disease Detection\\
	&	 26	&	 Alexander Hogenboom (EUR), Sentiment Analysis of Text Guided by Semantics and Structure\\
	&	 27	&	 S\'{a}ndor H\'{e}man (CWI), Updating compressed colomn stores\\
	&	 28	&	 Janet Bagorogoza (TiU), Knowledge Management and High Performance; The Uganda Financial Institutions Model for HPO\\
	&	 29	&	 Hendrik Baier (UM), Monte-Carlo Tree Search Enhancements for One-Player and Two-Player Domains\\
	&	 30	&	 Kiavash Bahreini (OU), Real-time Multimodal Emotion Recognition in E-Learning\\
	&	 31	&	 Yakup Ko\c{c} (TUD), On the robustness of Power Grids\\
	&	 32	&	 Jerome Gard (UL), Corporate Venture Management in SMEs\\
	&	 33	&	 Frederik Schadd (TUD), Ontology Mapping with Auxiliary Resources\\
	&	 34	&	 Victor de Graaf (UT), Gesocial Recommender Systems\\
	&	 35	&	 Jungxao Xu (TUD), Affective Body Language of Humanoid Robots: Perception and Effects in Human Robot Interaction\\

\midrule
2016&	 01	&	 Syed Saiden Abbas (RUN), Recognition of Shapes by Humans and Machines\\
	&	 02	&	 Michiel Christiaan Meulendijk (UU), Optimizing medication reviews through decision support: prescribing a better pill to swallow\\
	&	 03	&	 Maya Sappelli (RUN), Knowledge Work in Context: User Centered Knowledge Worker Support\\
	&	 04	&	 Laurens Rietveld (VU), Publishing and Consuming Linked Data\\
	&	 05	&	 Evgeny Sherkhonov (UVA), Expanded Acyclic Queries: Containment and an Application in Explaining Missing Answers\\
	&	 06	&	 Michel Wilson (TUD), Robust scheduling in an uncertain environment\\
	&	 07	&	 Jeroen de Man (VU), Measuring and modeling negative emotions for virtual training\\
	&	 08	&	 Matje van de Camp (TiU), A Link to the Past: Constructing Historical Social Networks from Unstructured Data\\
	&	 09	&	 Archana Nottamkandath (VU), Trusting Crowdsourced Information on Cultural Artefacts\\
	&	 10	&	 George Karafotias (VUA), Parameter Control for Evolutionary Algorithms\\
	&	 11	&	 Anne Schuth (UVA), Search Engines that Learn from Their Users\\
	&	 12	&	 Max Knobbout (UU), Logics for Modelling and Verifying Normative Multi-Agent Systems\\
	&	 13	&	 Nana Baah Gyan (VU), The Web, Speech Technologies and Rural Development in West Africa - An ICT4D Approach\\
	&	 14	&	 Ravi Khadka (UU), Revisiting Legacy Software System Modernization\\
	&	 15	&	 Steffen Michels (RUN), Hybrid Probabilistic Logics - Theoretical Aspects, Algorithms and Experiments\\
	&	 16	&	 Guangliang Li (UVA), Socially Intelligent Autonomous Agents that Learn from Human Reward\\
	&	 17	&	 Berend Weel (VU), Towards Embodied Evolution of Robot Organisms\\
	&	 18	&	 Albert Mero\~{n}o Pe\~{n}uela (VU), Refining Statistical Data on the Web\\
	&	 19	&	 Julia Efremova (Tu/e), Mining Social Structures from Genealogical Data\\
	&	 20	&	 Daan Odijk (UVA), Context \& Semantics in News \& Web Search\\
	&	 21	&	 Alejandro Moreno C\'{e}lleri (UT), From Traditional to Interactive Playspaces: Automatic Analysis of Player Behavior in the Interactive Tag Playground\\
	&	 22	&	 Grace Lewis (VU), Software Architecture Strategies for Cyber-Foraging Systems\\
	&	 23	&	 Fei Cai (UVA), Query Auto Completion in Information Retrieval\\
	&	 24	&	 Brend Wanders (UT), Repurposing and Probabilistic Integration of Data; An Iterative and data model independent approach\\
	&	 25	&	 Julia Kiseleva (TU/e), Using Contextual Information to Understand Searching and Browsing Behavior\\
	&	 26	&	 Dilhan Thilakarathne (VU), In or Out of Control: Exploring Computational Models to Study the Role of Human Awareness and Control in Behavioural Choices, with Applications in Aviation and Energy Management Domains\\
	&	 27	&	 Wen Li (TUD), Understanding Geo-spatial Information on Social Media\\
	&	 28	&	 Mingxin Zhang (TUD), Large-scale Agent-based Social Simulation - A study on epidemic prediction and control\\
	&	 29	&	 Nicolas H\"{o}ning (TUD), Peak reduction in decentralised electricity systems - Markets and prices for flexible planning\\
	&	 30	&	 Ruud Mattheij (UvT), The Eyes Have It\\
	&	 31	&	 Mohammad Khelghati (UT), Deep web content monitoring\\
	&	 32	&	 Eelco Vriezekolk (UT), Assessing Telecommunication Service Availability Risks for Crisis Organisations\\
	&	 33	&	 Peter Bloem (UVA), Single Sample Statistics, exercises in learning from just one example\\
	&	 34	&	 Dennis Schunselaar (TUE), Configurable Process Trees: Elicitation, Analysis, and Enactment\\
	&	 35	&	 Zhaochun Ren (UVA), Monitoring Social Media: Summarization, Classification and Recommendation\\
	&	 36	&	 Daphne Karreman (UT), Beyond R2D2: The design of nonverbal interaction behavior optimized for robot-specific morphologies\\
	&	 37	&	 Giovanni Sileno (UvA), Aligning Law and Action - a conceptual and computational inquiry\\
	&	 38	&	 Andrea Minuto (UT), Materials that Matter - Smart Materials meet Art \& Interaction Design\\
	&	 39	&	 Merijn Bruijnes (UT), Believable Suspect Agents; Response and Interpersonal Style Selection for an Artificial Suspect\\
	&	 40	&	 Christian Detweiler (TUD), Accounting for Values in Design\\
	&	 41	&	 Thomas King (TUD), Governing Governance: A Formal Framework for Analysing Institutional Design and Enactment Governance\\
	&	 42	&	 Spyros Martzoukos (UVA), Combinatorial and Compositional Aspects of Bilingual Aligned Corpora\\
	&	 43	&	 Saskia Koldijk (RUN), Context-Aware Support for Stress Self-Management: From Theory to Practice\\
	&	 44 &	 Thibault Sellam (UVA), Automatic Assistants for Database Exploration\\
	&	 45	&	 Bram van de Laar (UT), Experiencing Brain-Computer Interface Control\\
	&	 46	&	 Jorge Gallego Perez (UT), Robots to Make you Happy\\
	&	 47	&	 Christina Weber (UL), Real-time foresight - Preparedness for dynamic innovation networks\\
	&	 48	&	 Tanja Buttler (TUD), Collecting Lessons Learned\\
	&	 49	&	 Gleb Polevoy (TUD), Participation and Interaction in Projects. A Game-Theoretic Analysis\\
	&	 50	&	 Yan Wang (UVT), The Bridge of Dreams: Towards a Method for Operational Performance Alignment in IT-enabled Service Supply Chains\\

\midrule
2017&	 01	&	 Jan-Jaap Oerlemans (UL), Investigating Cybercrime\\
	&	 02	&	 Sjoerd Timmer (UU), Designing and Understanding Forensic Bayesian Networks using Argumentation\\
	&	 03	&	 Dani\"{e}l Harold Telgen (UU), Grid Manufacturing; A Cyber-Physical Approach with Autonomous Products and Reconfigurable Manufacturing Machines\\
	&	 04	&	 Mrunal Gawade (CWI), Multi-core Parallelism in a Column-store\\
	&	 05	&	 Mahdieh Shadi (UVA), Collaboration Behavior\\
	&	 06	&	 Damir Vandic (EUR), Intelligent Information Systems for Web Product Search\\
	&	 07	&	 Roel Bertens (UU), Insight in Information: from Abstract to Anomaly\\
	&	 08	& 	 Rob Konijn (VU)	, Detecting Interesting Differences:Data Mining in Health Insurance Data using Outlier Detection and Subgroup Discovery\\
	&	 09	&	 Dong Nguyen (UT), Text as Social and Cultural Data: A Computational Perspective on Variation in Text\\
	&	 10	&	 Robby van Delden (UT), (Steering) Interactive Play Behavior\\
	&	 11	&	 Florian Kunneman (RUN), Modelling patterns of time and emotion in Twitter \#anticipointment\\
	&	 12	&	 Sander Leemans (TUE), Robust Process Mining with Guarantees\\
 	&	 13	& 	 Gijs Huisman (UT),  Social Touch Technology - Extending the reach of social touch through haptic technology\\
 	&	 14	&	 Shoshannah Tekofsky (UvT), You Are Who You Play You Are: Modelling Player Traits from Video Game Behavior\\
	&	 15	&	 Peter Berck (RUN),  Memory-Based Text Correction\\
	&	 16	&	 Aleksandr Chuklin (UVA), Understanding and Modeling Users of Modern Search Engines\\
	&	 17	&	 Daniel Dimov (UL), Crowdsourced Online Dispute Resolution\\
	&	 18	&	 Ridho Reinanda (UVA), Entity Associations for Search\\
	&	 19	& 	 Jeroen Vuurens (UT), Proximity of Terms, Texts and Semantic Vectors in Information Retrieval\\
	&	 20	&	 Mohammadbashir Sedighi (TUD), Fostering Engagement in Knowledge Sharing: The Role of Perceived Benefits, Costs and Visibility\\
	&	 21	&	 Jeroen Linssen (UT), Meta Matters in Interactive Storytelling and Serious Gaming (A Play on Worlds)\\
	&	 22	&	 Sara Magliacane (VU), Logics for causal inference under uncertainty\\
	&	 23	&	 David Graus (UVA), Entities of Interest --- Discovery in Digital Traces\\
	&	 24	&	 Chang Wang (TUD), Use of Affordances for Efficient Robot Learning\\
	&	 25	&	 Veruska Zamborlini (VU), Knowledge Representation for Clinical Guidelines, with applications to Multimorbidity Analysis and Literature Search\\
	&	 26	&	 Merel Jung (UT), Socially intelligent robots that understand and respond to human touch\\
	&	 27	&	 Michiel Joosse (UT), Investigating Positioning and Gaze Behaviors of Social Robots: People's Preferences, Perceptions and Behaviors\\
	&	 28	&	 John Klein (VU), Architecture Practices for Complex Contexts\\
	&	 29	&	 Adel Alhuraibi (UvT), From IT-BusinessStrategic Alignment to Performance: A Moderated Mediation Model of Social Innovation, and Enterprise Governance of    IT"\\
	&	 30	&	 Wilma Latuny (UvT), The Power of Facial Expressions\\
	&	 31	&	 Ben Ruijl (UL), Advances in computational methods for QFT calculations\\
	&	 32	& 	 Thaer Samar (RUN), Access to and Retrievability of Content in Web Archives\\
	&	 33	&	 Brigit van Loggem (OU), Towards a Design Rationale for Software Documentation: A Model of Computer-Mediated Activity\\
	&	 34	&	 Maren Scheffel (OU), The Evaluation Framework for Learning Analytics \\
	&	 35	&	 Martine de Vos (VU), Interpreting natural science spreadsheets \\
	&	 36	&	 Yuanhao Guo (UL), Shape Analysis for Phenotype Characterisation from High-throughput Imaging \\
	&	 37	&	 Alejandro Montes Garcia (TUE), WiBAF: A Within Browser Adaptation Framework that Enables Control over Privacy \\
	&	 38	&	 Alex Kayal (TUD), Normative Social Applications \\
	&	 39	&	 Sara Ahmadi (RUN), Exploiting properties of the human auditory system and compressive sensing methods to increase   noise robustness in ASR \\
	&	 40	&	 Altaf Hussain Abro (VUA), Steer your Mind: Computational Exploration of Human Control in Relation to Emotions, Desires and Social Support For applications in human-aware support systems \\
	&	 41	&	 Adnan Manzoor (VUA), Minding a Healthy Lifestyle: An Exploration of Mental Processes and a Smart Environment to Provide Support for a Healthy Lifestyle\\
	&	 42	&	 Elena Sokolova (RUN), Causal discovery from mixed and missing data with applications on ADHD  datasets\\
	&	 43	&	 Maaike de Boer (RUN), Semantic Mapping in Video Retrieval\\
	&	 44	&	 Garm Lucassen (UU), Understanding User Stories - Computational Linguistics in Agile Requirements Engineering\\
	&	 45	&	 Bas Testerink	(UU), Decentralized Runtime Norm Enforcement\\
	&	 46	&	 Jan Schneider	(OU), Sensor-based Learning Support\\
	&	 47	&	 Jie Yang (TUD), Crowd Knowledge Creation Acceleration\\
	&	 48	&	 Angel Suarez (OU), Collaborative inquiry-based learning\\

\midrule
2018&	 01	&	 Han van der Aa (VUA), Comparing and Aligning Process Representations \\
	&	 02	&	 Felix Mannhardt (TUE), Multi-perspective Process Mining \\
	&	 03	&	 Steven Bosems (UT), Causal Models For Well-Being: Knowledge Modeling, Model-Driven Development of Context-Aware Applications, and Behavior Prediction\\
	&	 04	&	 Jordan Janeiro (TUD), Flexible Coordination Support for Diagnosis Teams in Data-Centric Engineering Tasks \\
	&	 05	&	 Hugo Huurdeman (UVA), Supporting the Complex Dynamics of the Information Seeking Process \\
	&	 06	&	 Dan Ionita (UT), Model-Driven Information Security Risk Assessment of Socio-Technical Systems \\
	&	 07	&	 Jieting Luo (UU), A formal account of opportunism in multi-agent systems \\
	&	 08	&	 Rick Smetsers (RUN), Advances in Model Learning for Software Systems \\
	&	 09	&	 Xu Xie	(TUD), Data Assimilation in Discrete Event Simulations \\
	&	 10	&	 Julienka Mollee (VUA), Moving forward: supporting physical activity behavior change through intelligent technology \\
	&	 11	&	 Mahdi Sargolzaei (UVA), Enabling Framework for Service-oriented Collaborative Networks \\
	&	 12	&	 Xixi Lu (TUE), Using behavioral context in process mining \\
	&	 13	&	 Seyed Amin Tabatabaei (VUA), Computing a Sustainable Future \\
	&	 14	&	 Bart Joosten (UVT), Detecting Social Signals with Spatiotemporal Gabor Filters \\
	&	 15	&	 Naser Davarzani (UM), Biomarker discovery in heart failure \\
	&	 16	&	 Jaebok Kim (UT), Automatic recognition of engagement and emotion in a group of children \\
	&	 17	&	 Jianpeng Zhang (TUE), On Graph Sample Clustering \\
	&	 18	& 	 Henriette Nakad (UL), De Notaris en Private Rechtspraak \\
	&	 19	&	 Minh Duc Pham (VUA), Emergent relational schemas for RDF \\
	&	 20	&	 Manxia Liu (RUN), Time and Bayesian Networks \\
	&	 21	&	 Aad Slootmaker (OUN), EMERGO: a generic platform for authoring and playing scenario-based serious games \\
	&	 22	&	 Eric Fernandes de Mello Araujo (VUA), Contagious: Modeling the Spread of Behaviours, Perceptions and Emotions in Social Networks \\
	&	 23	&	 Kim Schouten (EUR), Semantics-driven Aspect-Based Sentiment Analysis \\
	&	 24	&	 Jered Vroon (UT), Responsive Social Positioning Behaviour for Semi-Autonomous Telepresence Robots \\
	&	 25	&	 Riste Gligorov (VUA), Serious Games in Audio-Visual Collections \\
	&	 26	& 	 Roelof Anne Jelle de Vries (UT),Theory-Based and Tailor-Made: Motivational Messages for Behavior Change Technology \\
	&	 27	&	 Maikel Leemans (TUE), Hierarchical Process Mining for Scalable Software Analysis \\
	&	 28	&	 Christian Willemse (UT), Social Touch Technologies: How they feel and how they make you feel \\
	&	 29	&	 Yu Gu (UVT), Emotion Recognition from Mandarin Speech \\
	&	 30	&	 Wouter Beek,  The "K" in "semantic web" stands for "knowledge": scaling semantics to the web \\

\midrule
2019
	&	 01	&	 Rob van Eijk (UL),Web privacy measurement in real-time bidding systems. A graph-based approach to RTB system classification \\
	&	 02	&	 Emmanuelle Beauxis Aussalet (CWI, UU), Statistics and Visualizations for Assessing Class Size Uncertainty \\
	&	 03	&	 Eduardo Gonzalez Lopez de Murillas (TUE), Process Mining on Databases: Extracting Event Data from Real Life Data
				 Sources \\
	&	 04	&	 Ridho Rahmadi (RUN), Finding stable causal structures from clinical data \\
	& 	 05	&	 Sebastiaan van Zelst (TUE), Process Mining with Streaming Data \\
	&	 06	& 	 Chris Dijkshoorn (VU), Nichesourcing for Improving Access to Linked Cultural Heritage Datasets \\
	&	 07	&	 Soude Fazeli (TUD), Recommender Systems in Social Learning Platforms \\
	& 	 08	&	 Frits de Nijs (TUD), Resource-constrained Multi-agent Markov Decision Processes \\
	&	 09	&	 Fahimeh Alizadeh Moghaddam (UVA), Self-adaptation for energy efficiency in software systems \\
	&	 10	&	 Qing Chuan Ye (EUR), Multi-objective Optimization Methods for Allocation and Prediction \\
	&	 11	&	 Yue Zhao (TUD), Learning Analytics Technology to Understand Learner Behavioral Engagement in MOOCs \\
	&	 12	&	 Jacqueline Heinerman (VU), Better Together \\
	&	 13	&	 Guanliang Chen (TUD), MOOC Analytics: Learner Modeling and Content Generation \\
	&	 14	&	 Daniel Davis (TUD), Large-Scale Learning Analytics: Modeling Learner Behavior \& Improving Learning Outcomes in Massive Open Online Courses \\
	&	 15	&	 Erwin Walraven (TUD), Planning under Uncertainty in Constrained and Partially
				 Observable Environments \\
	&	 16	&	 Guangming Li (TUE), Process Mining based on Object-Centric Behavioral Constraint (OCBC) Models \\
	&	 17	&	 Ali Hurriyetoglu (RUN),Extracting actionable information from microtexts \\
	&	 18	&	 Gerard Wagenaar (UU), Artefacts in Agile Team Communication \\
	&	 19	&	 Vincent Koeman (TUD), Tools for Developing Cognitive Agents \\
	&	 20	&	 Chide Groenouwe (UU), Fostering technically augmented human collective intelligence \\
	&	 21	&	 Cong Liu (TUE), Software Data Analytics: Architectural Model Discovery and Design Pattern Detection \\
	&	 22	&	 Martin van den Berg (VU),Improving IT Decisions with Enterprise Architecture \\
	&	 23	&	 Qin Liu (TUD), Intelligent Control Systems: Learning, Interpreting, Verification\\
	&	 24	&	 Anca Dumitrache (VU),  Truth in Disagreement - Crowdsourcing Labeled Data for Natural Language Processing\\
	&	 25	&	 Emiel van Miltenburg (VU), Pragmatic factors in (automatic) image description \\
	&	 26	&	 Prince Singh (UT), An Integration Platform for Synchromodal Transport \\
	&	 27	&	 Alessandra Antonaci (OUN), The Gamification Design Process applied to (Massive) Open Online Courses\\
	&	 28	&	 Esther Kuindersma (UL), Cleared for take-off: Game-based learning to prepare airline pilots for critical situations \\
	&	 29	&	 Daniel Formolo (VU), Using virtual agents for simulation and training of social skills in safety-critical circumstances \\

    &    30 &    Vahid Yazdanpanah (UT), Multiagent Industrial Symbiosis Systems \\

    &    31 &    Milan Jelisavcic (VU), Alive and Kicking: Baby Steps in Robotics \\

    &    32 &    Chiara Sironi (UM), Monte-Carlo Tree Search for Artificial General Intelligence in Games \\

    &    33 &    Anil Yaman (TUE), Evolution of Biologically Inspired Learning in Artificial Neural Networks \\

    &    34 &    Negar Ahmadi (TUE), EEG Microstate and Functional Brain Network Features for Classification of Epilepsy
    and PNES  \\

    &    35 & Lisa Facey-Shaw (OUN), Gamification with digital badges in learning programming \\

    &    36 & Kevin Ackermans (OUN), Designing Video-Enhanced Rubrics to Master Complex Skills \\



    &    37 & Jian Fang (TUD),Database Acceleration on FPGAs \\

    & 38 & Ákos Kádár (TiU), Learning visually grounded and multilingual representations \\
\midrule
2020


    & 01 & Armon Toubman (UL), Calculated Moves: Generating Air Combat Behaviour \\



    & 02 & Macos de Paula Bueno (UL), Unraveling Temporal Processes using Probabilistic Graphical Models \\


    & 03 & Mostafa Deghani (UVA), Learning with Imperfect Supervision for Language Understanding \\


\bottomrule
\end{longtabu}

