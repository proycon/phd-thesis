\appendix*{Curriculum Vitae}

Maarten van Gompel was born on the \nth{15} of December 1982, in Etten-Leur,
the Netherlands. After completing his secondary education (VWO) at the
Heerbeeck College in Best in 2002, seeking for a study that combined his
passion for both language as well as technology, he enrolled in the study of
Cognitive Artificial Intelligence (CKI) at Utrecht University. He also
co-founded the UniLang Language Community, one of the first communities on the
internet for language enthusiasts and polyglots to unite and learn languages,
that thrived in the early 2000s before the advent of social media. He obtained
the degree of Bachelor of Science from Utrecht University in 2008 and then
switched to Tilburg University for his master degree in "Human Aspects of
Information Technology", looking for a stronger focus on Natural Language
Processing (NLP).  This resulted in the degree of Master of Arts in 2009.

He stayed in Tilburg at the Induction of Linguistic Knowledge (ILK) research
group, led by prof. dr. Antal van den Bosch, for the next two years working on
the DutchSemCor project as a scientific programmer. Around this time, he also
started getting involved in the CLARIN project and started two software
initiatives, CLAM (a tool to quickly create webservices out of NLP tools) and
FoLiA (a Format for Linguistic Annotation) that are still in active development
to this day, a decade later, and provided a foundation for a lot of his later work.

In 2011, Antal van den Bosch and Maarten van Gompel transferred to Radboud
University , Nijmegen, where the PhD research started. The research group that
formed over the years came to be known as the "Lamas", short for "Language
Machines".  Alongside his work on this PhD research, Maarten has been actively
working as a research software engineer. As an avid open-source and free
software proponent, he is involved in initiatives promoting good software
quality \& sustainability practices in the research community. He created and
maintains various software tools, including the Colibri Core software that
culminated from his PhD project.

Though the formal conclusion of the PhD research was delayed until 2020, the
actual research already reached its conclusion in 2016, and ever since he has
been working on software development for, amongst others, the CLARIAH and
Nederlab project, as well as performing duties in linux system administration.
