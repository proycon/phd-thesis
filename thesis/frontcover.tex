\documentclass[11pt]{article}
%width =  pagewidth * 2  + (pagethickness*pages/2) + bleed*2
%width = 170 * 2 + (0.108*213/2) + 0*2 =
\usepackage[paperwidth=170mm,paperheight=240mm,margin=0mm]{geometry}
\usepackage[dvipsnames,prologue,table]{pstricks}
\usepackage{pst-text}
\usepackage{pst-grad}
\usepackage{graphicx}


\begin{document}
\pagestyle{empty}

\definecolor{ferngreen}{rgb}{0.31,0.47,0.26}
\definecolor{greenwhite}{rgb}{0.68,0.78,0.67}

\newsavebox\IBox
\sbox\IBox{\includegraphics[width=130mm]{drawing.eps}}

%set up the picture environment
\psset{unit=1mm}\noindent
\begin{pspicture}(170mm,240mm)
% set up the fonts we use


\DeclareFixedFont{\PT}{T1}{ppl}{b}{n}{24pt}
\DeclareFixedFont{\PTsmall}{T1}{ppl}{b}{it}{20pt}
\DeclareFixedFont{\PTauthor}{T1}{ppl}{n}{n}{18pt}
\DeclareFixedFont{\PTsmallest}{T1}{ppl}{b}{it}{0.3in}
\DeclareFixedFont{\PTtext}{T1}{ppl}{b}{it}{11pt}
\DeclareFixedFont{\Logo}{T1}{pbk}{m}{n}{0.3in}

% create a background
\psframe[fillstyle=solid,fillcolor=ferngreen,linecolor=ferngreen](0,0)(170,240)

% white box for the drawing
\psframe[fillstyle=solid,fillcolor=white,linecolor=white](0,80)(170,240)

% place the front cover picture
\rput[lb](20,100){\usebox\IBox}
% put the text on the front cover
\rput[lb](19,60){\PT \color{white}{Co}\color{greenwhite}{nstructions} \color{greenwhite}{as} \color{white}{Li}\color{greenwhite}{nguistic} \color{white}{Bri}\color{greenwhite}{dges}}
\rput[lb](96,50){\PTauthor \color{white}{Maarten van Gompel}}
%\rput[lb](239,15){\PTsmallest \color{black}{Lughdunum Press}}


% Then we close all open environments

\end{pspicture}

\newpage
\pagestyle{empty}
\psset{unit=1mm}\noindent
\begin{pspicture}(170mm,240mm)
\psframe[fillstyle=solid,fillcolor=ferngreen,linecolor=ferngreen](0,0)(170,240)
% Create a Box containing the text for the back cover
\newsavebox\Summarybox
\sbox\Summarybox{\begin{minipage}{85mm}

{
\textcolor{white}{\textbf{Context as Linguistic Bridges}} \textcolor{greenwhite}{ is a study that focusses on the role of context information in machine translation, i.e.
automated translation by computers.  The underlying intuition is that the context in which a word or phrase appears is
an important cue for the translation of that word or phrase. Consider, for example, the two different meanings of the word
\emph{``bank''} in the sentences \emph{``I put my money on the bank''} and \emph{``The ship got stuck on the bank''}.
}

\vspace{5mm}

\textcolor{greenwhite}{
We developed classifier-based solutions that work well in Word Sense Disambiguation tasks like the above example, and
integrate these in a Statistical Machine Translation system.  Our main question is to find to what extent can we improve
automated translation by explicitly modelling such context information.
}

}
\end{minipage}}

% And position the box
\rput[tl](42.5,213){\usebox\Summarybox}
\end{pspicture}
\end{document}
