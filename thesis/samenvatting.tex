\chapter*{Samenvatting}
\addcontentsline{toc}{chapter}{Samenvatting}

Dit proefschrift, getiteld \emph{Context as Linguistic Bridges}, richt zich op het onderzoeken van de rol van
context in automatische vertaalsystemen. Het onderliggende idee is dat de context waarin een woord of zinssnede zich
bevindt belangrijke informatie geeft voor de vertaling van dat woord of die zinssnede; het helpt om ambiguïteit op te
lossen.
Neem, bijvoorbeeld, het Nederlandse woord ``bank'', dit kan een financieel instituut betekenen of een bank om op te
ziten. In zinnen zoals ``Ik zet mijn geld op de bank'' en ``De hond mag niet op de bank'', helpt de context ons om de
juiste betekenis te identificeren. Dit wordt, met een engelse term, ook wel Word Sense Disambiguation (WSD) genoemd. In
dit proefschrift onderzoeken we technieken om dit te doen op een automatische manier. Vervolgens proberen we deze
oplossingen te integreren in een automatische vertaalsysteem. Het is namelijk zo dat als we deze woorden zouden vertalen
naar bijvoorbeeld het Engels, we een andere vertaling krijgen bij een andere betekenis. In dit voorbeeld zouden we
respectievelijk ``bank'' en ``sofa'' of ``couch'' krijgen.

We onderzoeken twee bepaalde technieken om twee verschillende problemen op te lossen en proberen deze dan te
combineren: Geheugen-gebaseerde classificatiesystemen blijken succes te hebben bij Word Sense Disambiguation, en
frase-gebaseerde statistische vertaalsystemen waren ten tijde dat ons onderzoek begon de beste oplossing voor
automatische vertaling (maar zijn inmiddels gepasseerd door diepe neurale netwerken). Deze twee technieken combineren we
om onze hoofdvraag te beantwoorden:

\begin{quote}
In hoeverre kunnen we automatische vertaling verbeteren door contextinformatie uit de brontaal mee te nemen?
\end{quote}

Dit onderzoek slaat ook nog een extra zijspoor in door een apart soort vertaalprobleem te beschouwen, namelijk het vertalen
van een woord of zinssnede in een anderstalige context. Dit is fenomeen wat ook wel als codewisseling bekend staat. Neem
bijvoorbeeld een fransman die Nederlands (L2) probeert te spreken maar niet geheel de juiste woorden kan vinden en
daarom voor een deel terugvalt op zijn moedertaal (L1): ``Ik ga .. eh.. \emph{rentre à la maison}... omdat ik moe ben''.
We proberen dit L1 fragment in een L2 context automatisch naar L2 om te zetten.

In onze zoektocht naar een antwoord op de hoofdvraag, komen er ook een aantal gerelateerde vragen aan bod:

\begin{itemize}
\item \emph{Welke contextinformatie is het meest effectief?} - Welke informatie uit de context in de brontaal draagt het
    meeste bij voor een juiste vertaling?
\item \emph{In hoeverre is niet-linguïstisch geinformeerde informatie voldoende?} - We willen zo dicht mogelijk bij de
    brontekst bljiven als mogelijk, zonder al te veel extra linguïstische informatie zoals woordsoort en lemma toe te
    hoeven voegen. We onderzoeken wel wat voor een verschil bepaalde soorten linguïstische verrijking uitmaakt.
\item \emph{Hoe kunnen classificatiesystemen geintegreerd worden in automatische vertaalsystemen?} - Er zijn
    verschillende manieren waarop deze integratie gedaan kan worden en verschillende keuzes te maken, we onderzoeken er een aantal.
\end{itemize}

We moeten uiteindelijk concluderen dat de integratie van context-gevoelige classificatiesystemen in een automatisch
vertaalsysteem niet leidt tot enige significante verbetering, in tegenstelling tot onze aanvankelijke hypothese. Dit
doet an sich geen afbreuk aan het feit dat contextinformatie in de brontaal nuttige informatie oplevert voor
classificatie, zoals we in ons WSD onderzoek aantonen, maar het betekent dat het expliciet modelleren van deze
informatie in een automatisch vertaalsysteem geen meerwaarde ten opzichte van de al beschikbare vertaalmodellen. We
leiden daaruit af dat er al voldoende van deze informatie impliciet beschikbaar is deze bestaande modellen.

