\chapter{Introduction}

“To know an object is to lead to it through a context which the world
provides.” -- William James (American Philospher, 1861-1947)

"Form is emptiness, emptiness is form" -- Heart Sutra

To even begin to merit the title of ``Doctor of Philosophy'', it is only proper
to start this dissertation with some philosophical deliberations. This will be
in sharp contrast to highly technical nature of the rest of these chapters.

The study you are about to read sprung from the intuition that \emph{context}
is an important, if not the most important, characteristic that defines
language. Language itself only exists only in the context of the world that
surrounds us, as well our internal mental world. Without this context, there is
nothing to talk about in the first place. Language is rightfully considered the
epitome of human evolution. Our species evolved the remarkable ability to refer to the
world outside and within by producing complex meaningful utterances, i.e. speech. 

This was an unparalleled revolution in \emph{communication}, which has
undoubtedly played a leading role in us becoming the dominant intelligent species
on the planet. It has put us in a position where we can communicate our
thoughts and feelings about the world to one another on a more fine-grained
level than any other species can. Moreover, it has given us the ability of ever increasing
\emph{abstraction}. The context of our language is no longer limited to just
refer to objects in our immediate vicinity, but we can even refer to abstract
thought itself. Whenever communication is attempted between people who do not
share a language, the context to which can be referred is dramatically restricted.
Just imagine tourists in a foreign bakery lustfully pointing at the
pastries they desire. 

Another major revolution in communication has been the development of writing,
and much later that of print technology. This and the accompanied literacy of
populations allows for our thoughts and ideas to be preserved more easily and
accurately than oral traditions can accomplish. The ability to read and write
broadens one's world, one's context, and is therefore even deemed a fundamental
human right. 

Language is inherently ambiguous and context is the disambiguating factor
without which it can not exist. The context of a bakery and a neatly arranged
line of pastries is essential for the baker to be able to disambiguate the
pointing of the clueless foreign tourist and discern which pastry he actually
wants. Demonstrative pronouns such as ``he'', ``she'', ``this'', or even
definite noun phrases such as ``the house'' convey little information if not
for the context they're employed in.  In fact, any word by itself is pretty
limited, can never exist in total isolation, and only derives meaning from the
further context. It is even defined purely through the context, just like a
dictionary defines words in terms of other words.

Buddhist philosophy has a concept called SUNYATA. Sorry? You couldn't decipher
these cute curly snakes hanging from a line? If the context of your own
upbringing was Indian and you had been brought up learning how to read
devanagari script, you would probably be able to read that as ``shunyata''.
Merely being able to read and vocalize the word still wouldn't bring you closer
to its meaning though. The word is Sanskrit, though it comes from a Pali word,
as is the form in which it is written in ancient Buddhist scriptures. The fact
we can can even still talk about this word and its usage in a 2500-year old
text today is already a testimony to the major impact writing technology has on
our civilizations, but that is not my main point here. Sunyata can be
translated into English as ``emptiness'' and refers to the idea that everything
is inter-related and nothing has eny existence or essence by itself in
isolation: everything exists and is defined purely through its context.

Having established the importance of context, and now leaving both sunyata and
pastries aside, we can dive into the actual subject matter.

\section{Linguistic Bridges}





