\chapter{Introduction}

“To know an object is to lead to it through a context which the world
provides.” -- William James (American Philospher, 1861-1947)

"Form is emptiness, emptiness is form" -- Heart Sutra

\section{Context as Linguistic Bridges}

\subsection{Context}

To even begin to merit the title of ``Doctor of Philosophy'', it is only proper
to start this dissertation with some philosophical deliberations. This will be
in sharp contrast to more technical nature of the rest of these chapters.

The study you are about to read sprung from the intuition that \emph{context}
is an important, if not the most important, characteristic that defines
language. Language itself only exists only in the context of the world that
surrounds us, as well our internal mental world. Without this context, there is
nothing to talk about in the first place. Language is rightfully considered the
epitome of human evolution. Our species evolved the remarkable ability to refer to the
world outside and within by producing complex meaningful utterances, i.e. speech. 

This was an unparalleled revolution in \emph{communication}, which has
undoubtedly played a leading role in us becoming the dominant intelligent species
on the planet. It has put us in a position where we can communicate our
thoughts and feelings about the world to one another on a more fine-grained
level than any other species can. Moreover, it has given us the ability of ever increasing
\emph{abstraction}. The context of our language is no longer limited to just
refer to objects in our immediate vicinity, but we can even refer to abstract
thought itself. Whenever communication is attempted between people who do not
share a language, the context to which can be referred is dramatically restricted.
Just imagine tourists in a foreign bakery lustfully pointing at the
pastries they desire. 

Another major revolution in communication has been the development of writing,
and much later that of print technology. This and the accompanied literacy of
populations allows for our thoughts and ideas to be preserved more easily and
accurately than oral traditions can accomplish. The ability to read and write
broadens one's world, one's context, and is therefore even deemed a fundamental
human right. 

Language is inherently ambiguous and context is the disambiguating factor
without which it can not exist. The context of a bakery and a neatly arranged
line of pastries is essential for the baker to be able to disambiguate the
pointing of the clueless foreign tourist and discern which pastry he actually
wants. Demonstrative pronouns such as ``he'', ``she'', ``this'', or even
definite noun phrases such as ``the house'' convey little information if not
for the context they're employed in.  In fact, any word by itself is pretty
limited, can never exist in total isolation, and only derives meaning from the
further context. It is even defined purely through the context, just like a
dictionary defines words in terms of other words.

Buddhist philosophy has a concept called शून्यता. Sorry? You couldn't decipher
these cute curly snakes hanging from a line? If the context of your own
upbringing was Indian and you had been brought up learning how to read
devanagari script, you would probably be able to read that as ``shunyata''.
Merely being able to read and vocalize the word still wouldn't bring you closer
to its meaning though. The word is Sanskrit, though it comes from a Pali word,
as is the form in which it is written in an Buddhist scripture called the
``Heart Sutra''. 

The fact we can can even still talk about this word and its usage in a
2500-year old text today is already a testimony to the major impact writing
technology has on our civilization. Hundreds of millions of adherents of
various major religions still derive their daily religious practice from
throughts and ideas put down in writing in ancient times. Moreover, as a
scientific community, we base our studies on the work of our predecessors,
explicitly citing them as our methodology requires. But I digress, as my
main point relates to the meaning of sunyata. 

Sunyata can be translated into English as ``emptiness'' and refers to the idea
that everything, not just words, is interrelated and nothing has any existence
or essence of or by itself: this philosophy posits that everything exists and
is defined purely through \emph{its context}.

\subsection{Linguistic Bridges}

Whereas we humans all share an astounding capacity for language, history has
given rise to numerous distinct and often mutually unintelligable languages.
This bring us back to the predicament of our pastry-loving tourist in the
foreign bakery, unable to express his choice in the language of the baker and
therefore resorting to pointing. If we situate the bakery in France, the baker
may reply ``croissant'' in response to the gesturing of his client. The
perceptive client may then have learned that this is what the pastry is called.
The famous utterance ``Me Tarzan, You Jane'', heavily supported by gestures as in
the classic movie ``The Ape Man'', is of a similar nature. Both protagonists
breach the language barrier and gain new information. These example show that context
plays an important role in establishing translation, or any common vocabulary. 

The \emph{linguistic bridges} of this dissertation refer to these acts of
translation. We can define translation as \emph{a process that yields a
representation of a message initially expressed in one language, in another}.
Etymologically, the word \emph{translation}, from the latin \emph{translātiō},
refers to carrying (lātiō) accross (trans) something. The focus is generally on
accurate preservation of the semantics of message.  Although in some arts, such
as poetry, form or emotion may take presendence over the substance.

The intuition underlying our research is that the context of word or phrase is
an important cue for the translation of that word or phrase. A word in context
A, may be translated differently than the same word in context B. Consider the
Portuguese word ``tempo`` in the following sentences:

\begin{enumerate}
\item ``O \textbf{tempo} é bom hoje.'' -- ``The \textbf{weather} is nice today.''
\item ``Não tenho \textbf{tempo} para estudar hoje.'' -- ``I don't have \textbf{time} to study today.''
\end{enumerate}

In the first sentence, ``tempo`` is translated as ``weather'', whereas in the
second it is translated as ``time''. Portuguese uses the same word where
English uses two distinct words; context provides cues to disambiguate into the
proper translation. Another example shows where English has one word and French has two:

\begin{enumerate}
\item ``J'ai tué \textbf{la mouche}'' -- ``I killed \textbf{the fly}''
\item ``Je \textbf{vole} à Paris'' - ``I \textbf{fly} to Paris''
\item ``Je \textbf{vole} l'horloge de mon père'' -- ``I steal my father's watch''
\end{enumerate}

In this latter example, disambiguation between the first two sentences is
facilitated by the fact that the word \emph{fly} has a different
part-of-speech in both sentences, unlike the noun \emph{tempo} in the earlier example.
The presence of the article ``the'' in the noun phrase ``the fly'' already
rules out a translation to a verb. The French verb ``voler'' in turn does not
just translate to ``to fly'', but can also mean ``to steal'', as shown in
the last sentence. 

These examples illustrate that languages are almost never translatable on a naive
word-by-word basis, and that context plays an major role in determining the
right translation. Context plays a bridging function in translation, and this
dissertation investigates techniques for machines to exploit this to attain
better automatic translations.

\subsection{Natural Language Processing}

In our philosophical deliberations we briefly addressed the communication
revolutions brought about by humanity evolving the faculty of language and
speech, and later the technological development of writing and print. We all
live in fortunate times to have just witnessed the biggest revolution in
communication since the invention of print: the information revolution. Text is
not just printed anymore, it is digitized; i.e. made available, distributable,
and searchable in digital form. The internet enables these digital resources to
be collected, shared, and exploited at unpredecented rate and allows us to
communicate instantaneously with people from all over the planet. 

To emphasize the importance of this development, and to underline my personal
conviction that this is the way the future should go; this dissertation itself will
be made available primarily in an open digital form, rather than printed book form.

This wealth of digitized data fuels our field of research: Natural Language
Processing (NLP). We attempt to extract meaningful data from natural language.
Most contemporary techniques in NLP are grounded in Machine Learning, which
employs statistical principles to learn whatever patterns from big collections
of natural language data. These data-driven approaches in NLP can be contrasted to
the more rule-based approaches, not without their own merit, that are directly
derived from human expert knowledge.

This dissertation strictly follows the data-driven trend and we therefore rely
on large amounts of digitized text to conduct our research.


%%

\subsection{}

To investigate the role of context information in translation, our
begins in a field that shares a 






\subsection{Machine Translation}

The area of research dedicated to the automatic translation of data from one
natural language to another is called \emph{Machine Translation} (MT).
State-of-the-art techniques in this field are primarily based on statistical
methods, in which case we can speak of \emph{Statistical Machine Translation}
(SMT). 

Early methods in Machine Translation 



\section{Thesis structure}























