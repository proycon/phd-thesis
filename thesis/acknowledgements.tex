\chapter*{Acknowledgements}
\addcontentsline{toc}{chapter}{Acknowledgements}
\markboth{Acknowledgements}{Acknowledgements}

Thanks first and foremost goes to Antal van den Bosch, with whom I moved from
Tilburg University to Radboud University to start this PhD research and who has
been my supervisor throughout, as well as co-authored many papers. Together we formed
the ideas and intuitions that drove this research, adjusting our direction as
we went along. Our research together already started in 2009
and culminated in my Master's thesis at that time.

Two years later, just after summer 2011, we stood at the dawn of the
development of a new research group in Nijmegen that would later come to be
known as the ``Language Machines'', or affectionately called ``the lamas''.  I
am happy to have been present since its inception, and happy to see
various familiar faces from our previous research group join us, as well as
meet many new colleagues who joined us. On the flip side, I am sad to see various colleagues
move on and the group shrinking considerably these past few years, but such is life.

I am immensely grateful for the confidence and freedom I received throughout the
years, also from Henk van den Heuvel of the Centre of Language and Speech
Technology (CLST), to work primarily from home and forego on any far travels,
which my health situation prohibits unfortunately. I have had the opportunity to take
initiative in starting numerous research software development projects. Many of
those go under the umbrella of CLARIN-NL and later CLARIAH; to name a few:
CLAM, FoLiA, FLAT, LaMachine, Ucto... In the light of the combined weight of
all this research software, I even consider the current work that lies before
you relatively minor, notwithstanding the effort that has gone into it and the
great opportunity it has been.

I want to thank the co-authors of Chapter~\ref{chap:semeval2014task5}, Iris
Hendrickx and Els Lefever, without whom the shared task for SemEval that we
describe in that chapter would not have been possible. Similarly, thanks also
goes to the participants in said shared task. I consider this line of
investigation one of the more succesful outcomes of this PhD project.

Finishing the dissertation was not without its challenges. The research itself
had already finished in early 2016 but it was apparent that it became largely a
negative result report rather than the breakthrough one initially hopes for.
This was also cause for some rejections from publishers, regarding
Chapters~\ref{chap:colibritafinal} and \ref{chap:sourcecontextinsmt}. In the
meantime the various other research software development projects proved quite
succesful and took up so much time that there was little time, and less
motivation, left to actually finish the dissertation. Thanks to CLST funding I
was eventually able to take some time to stop other activities and direct
all focus to completion of the dissertation and its defense.

Thanks to my sister Elke for drawing the beautiful hummingbird that you see on the cover of this book.  We have a bit of
a tradition to name our software after animals, so I wanted to continue in this tradition for the software that sprung
from this research (see Chapters~\ref{chap:coco} and \ref{chap:colibritafinal}), as well as for the project as a whole.
Colibri, or some variation thereof, is the word for hummingbird in many languages, and is an acronym of the somewhat
cryptic title of this dissertation: \emph{Constructions as Linguistic Bridges}.

Last, but certainly not least, thanks goes to my boyfriend Hans for his endless patience in putting up with
me (and no less important; feeding me!) when I often work strange hours and zone
out as I am emerged in some work-related project, whether it is programming or
writing this dissertation, or even chatting with colleagues on our IRC chat.
The latter has proven a most cherished tool that allows me to keep in touch
with my colleagues no matter the time or place!

