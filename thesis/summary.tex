\chapter*{Summary}
\addcontentsline{toc}{chapter}{Summary}
\markboth{Summary}{Summary}

\emph{Context as Linguistic Bridges} is a study that focusses on the role of
context information in machine translation, i.e. automated translation by computers.  The
underlying intuition is that the context in which a word or phrase appears is
an important cue for the translation of that word or phrase; it helps in
resolving ambiguity.

Consider, for instance, the English word \emph{``bank''}, which can refer to a
financial institution or the bank of a river. In sentences such as \emph{``I
put my money on the bank''} and \emph{``The ship got stuck on the bank''}, the
context helps to disambiguate the word. Automatically identifying what the corredct sense of a word is, is known as Word Sense
Disambiguation (WSD). In this research we develop techniques to do this
in and attempt to apply these same techniques in a Machine
Translation context. After all, if we were to translate these two distinct
word senses to another language, we would likely obtain different words. For
Dutch we would get \emph{``bank''} and \emph{``oever''}, respectively.

We look into two particular solutions to solve two different problems and then
try to combine these into one: \emph{Memory-based Classification} proves
successful in Word Sense Disambiguation tasks, and \emph{Phrase-based
Statistical Machine Translation} used to be the state-of-the-art in machine
translation when this research began (but it has been superseded since by deep
learning). We combine these two techniques and attempt to answer the following main
question:

\begin{quote}
\emph{To what extent can we improve translation by considering source-side context information?}
\end{quote}

In this sense, ``source-side context'' refers to the fact that we use
information from the input in the source language, which is to be translated
into the target language.

We also add a twist to this research as we investigate a special type of
translation using source-side context; the translation of a word or phrase in a
cross-lingual context; a phenomenon also known as code-switching. Consider a
frenchman attempting to speak English (L2), but not finding quite the right
words for everything so reverting back to his native language (L1) for that
fragment: ``I go ...eh.... \emph{rentre à la maison}... because I am tired.''
. We try to automatically translate this L1 fragment in an L2 context to L2.

In our search for an answer to the main question, we can distinguish three subquestions that will be addressed:

\begin{itemize}
\item \emph{What context features prove most effective?} - What exactly in the source-side context provides the most valuable cue for a succesful disambiguation?
\item \emph{To what extent are linguistically uninformed features effective?} -  Our aim is to stay as close to the pure
    textual data as possible, not aided by further linguistic information such as part-of-speech tags or lemmas. We do
    conduct various experiments to see what difference such linguistic enrichment makes.
\item \emph{How can source-side classifiers be used in translation tasks?} - How exactly do we integrate classifiers inside a phrase-based statistical machine translation system? There are various possibilities and choices to make and we investigate and compare several.
\end{itemize}

We have to conclude that the integration of context-aware classifiers in a
phrase-based statistical machine translation system, does not lead to any
significant improvement, contrary to our initial hypothesis. This does not
discredit the idea that source-side context provides valuable cues for
translation or disambiguation, as we show in our WSD research.  It shows only
that modelling this source-side context information explicitly provides no
extra benefit over the already existing statistical machine translation model, and we infer
that the information is already available implicitly to a sufficient degree in the
existing statistical machine translation model.

